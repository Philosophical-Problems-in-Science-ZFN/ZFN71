\begin{artengenv}{Marcin Gorazda}
	{Can we remain rational in the large world? On some unexpected consequences of ecological rationality}
	{Can we remain rational in the large world?\ldots}
	{Can we remain rational in the large world? On some unexpected\\consequences of ecological rationality}
	{Copernicus Center for Interdisciplinary Studies}
	{The paper outlines various concepts of rationality, their characteristics and consequences. In the first, most general part, the metaphysical, instrumental and discursive rationality is distinguished. The following part focuses on instrumental rationality and the rational choice theory and ordinal and cardinal utility, expected utility and game theory, respectively. All those concepts are summarised as being the most mathematically elegant and mostly decidable and helpful in the decision-making process. Giving primacy to individual preferences and withholding the judgment on their ``objective'' value, they are also devoid of double standards. They are, however, strongly normative and weakly coincide with actual agents' behaviour. Empirical findings on agents' decision making seem to demonstrate their irrationality, unless we introduce into the analysis different concepts of rationality, namely based on costs efficient heuristics, inclusive fitness and ecological rationality. They are discussed respectively, and although they seem better to explain the set of humans' seemingly irrational behaviour, they are likely week in predicting that behaviour. They are also losing their normative dimension and thus cease to be helpful in decision making. Applying the particular theory of rationality, either descriptively or normatively, seems to depend strongly on the environment, which can be characterised by its extension from a~small to a~large world. The more the small world's features an environment reveals, the more effective is the application of the particular model of rationality. Beyond the small worlds, rule stochasticity, underspecification and misspecification and the only reasonable method are consecutive trials and errors, which eventually may reduce the large world to the small one.}
	{rationality, rational choice theory, ecological rationality, inclusive fitness, decision theory.}




\section*{Introduction}
\lettrine[loversize=0.13,lines=2,lraise=-0.01,nindent=0em,findent=0.2pt]%
{S}{}o long as we can trackback humans' reflections on life, one of the most important dilemmas was making a~good decision and what it means ``good'' in principle. Although the dilemma has never been decisively resolved (and probably will never be), at the very beginning, those who pondered over it were at least able to indicate the examples of ``good'' and ``bad'' decisions. They often concluded that those ``better'' were made deliberately (not impulsively or emotionally), with the engagement of our reason (lat. \textit{ratio}). They were rational. Reasoning solely does not suffice. It needs instructions, how to reason reasonably or rationally. Here begins the story of the ambiguity of the term rationality. B.Brożek rightly notes that the problem of rationality may be inherently unresolvable, as ``setting the criterion of rationality requires a~prior knowledge of how to do this rationally''
%\label{ref:RNDOuLEr1sSoh}(Brożek, 2007, p.158).
\parencite[][p.158]{brozek_rationality_2007}. %
 So we find ourselves in a~vicious circle with no way out. It has not deterred, however, the philosophers to undertake the efforts. So, in the philosophical literature, we have plenty of concepts of rationality and thus plenty of definitions to which we can refer. Although the problem has never been resolved, there are signs of particular progress, or at least a~shift in analysis, which goes in line with the general approach to many other philosophical problems. At the beginning of known considerations, rationality was placed in the abstract, transcendent realm and conceived as the objective, unshakable patterns of humans' behaviour, which must be correctly recognised and obeyed. It had strong normative connotations. It did not matter how humans actually behave or justify their decisions. It only mattered how they ought to behave. The shift, which may be called epistemological-behavioural, was towards the comprehensive reflection on how humans recognise and internalise those ``patterns'' and how they actually behave and are in harmony with the progress in cognitive psychology and evolutionary theory. This paper is about the shift from ``metaphysical'' to ``ecological'' rationality and on some consequences on the latter, which raises the question of limits of our rationality.

\section*{Three different concepts of rationality}
In antiquity and the Middle Ages, philosophers were bound to the concept of metaphysical rationality. The world was perceived as a~highly ordered structure, governed by the predefined rules, usually of divine origin, relatively stable and cognisable. \textit{Logos} encompassed both natural and moral laws, which were indistinguishable by their nature. The principles of rationality belonged to this transcendent sphere. To decide rationally meant to recognise, be aware and follow those principles. Even if they were sometimes perceived as changeable, the change itself was embedded into the order and had its reasons. This mode of thinking dominated over centuries, at least until the early modern times, and is still vivid in some branches of Christian philosophy, where often to be rational means to fulfil God's will or where the principles of reason are engaged to justify the rationality of religious beliefs
%\label{ref:RNDOPBRAKDSOB}(Wszołek, 2004; Jordan, 2006; Gorazda, 2009).
\parencites[][]{wszolek_wprowadzenie_2004}[][]{jordan_pascals_2006}[][]{gorazda_pragmatyzm_2009}. %
 The most important problem usually pointed out at the metaphysical rationality is the problem of rational principles cognition. Even if one assumes that there is an ontologically ordered structure, taking into account numerous theories of that structure, one can hardly find a~reliable method to recognise it and even harder the principles of rationality within it. The decision of those principles admission is in fact, an act of belief combined with the previous ascription of credibility to specific human authority, responsible for the proper ``transmission'' of the hidden, ontological knowledge to her disciples.

Due to those dubious metaphysical roots, instrumental rationality became dominant at the beginning of the 19\textsuperscript{th} century, although some traces can be found even earlier. In this approach, the rational action is determined by the desired end. One abstracts from the judgement on the end itself, which is placed beyond the interest of the judge and belongs to the private sphere of the acting agent, focusing instead on the appropriate means to achieve an end. The early consideration in this spirit gave rise to the foundation of the decision theory. One of the most outstanding examples was the so-called ``Pascal's wager,'' where the eschatological ends were analysed instrumentally, and those considerations led to the construction of the proto-decision-matrix
%\label{ref:RNDQSEpoyoXCD}(Jordan, 2006).
\parencite[][]{jordan_pascals_2006}. %
 The ``means-ends'' concept of reason was developed and influenced American pragmatism and legal realism strongly, J.Dewey being the one who proposed and defended the idea of consequentialists logic as a~foundation of social action 
%\label{ref:RND1ixzED0a4q}(Dewey, 1924).
\parencite[][]{dewey_logical_1924}. %
 Legal realists soon discovered the fundamental problem with consequentialists logic in law-making---the problem with ends-defining. Without the reference to any metaphysical ideals (God's law, natural law or any other) or at least to any commonly shared values, we are blundering in the ethical vacuum without good reasons to justify any proposed end to be followed. The ends proposed by the ruling authority usually fulfilled a~concept of what is worth pursuing, but not precisely the commonly approved concept 
%\label{ref:RNDRYpfMAvsfz}(Pound, 2006; Gorazda, 2017).
\parencites[][]{pound_social_2006}[][]{gorazda_law_2017}.%


The third concept (or rather a~group of various concepts) of rationality is an attempt to face those two problems, namely the problem of metaphysical decidability or ``transferability'' and the problem of ``ideological vacuum.'' It is discursive or communicational rationality, which emerged in the 20\textsuperscript{th} century and can be briefly described as referring to the criteria of rationality acceptable in the idealised discussion. Brożek finds the sources of this rationality in Kant's categorical imperative, although in contemporary philosophy, the better representative is R. Alexy
%\label{ref:RNDuMzJbQXXP3}(1992).
\parencite*[][]{alexy_discourse-theoretical_1992}. %
 The main feature which distinguishes this concept is the reference to the chosen form of universality. In the case of Kant, these are ``principles which can be conceived and willed as universal law'' 
%\label{ref:RNDxvisyrudS9}(Brożek, 2007, p.170).
\parencite[][p.170]{brozek_rationality_2007}. %
 In the case of Alexy the universality originates from the ``conditions of rational, practical argumentation [...], system of rules of discourse.'' 
%\label{ref:RNDNTaIo7btik}(Alexy, 1992, p.235).
\parencite[][p.235]{alexy_discourse-theoretical_1992}. %
 Alexy is quite specific in formulating those rules and puts among them the principle of truth (compliance with beliefs), three Rationality Rules (common and open access to the discourse, liberty of speech, no coercion), Rules for Allocating the Burden of Proof and Rule of Justification. Whenever the rules are met, the attendees are capable of reaching the universal agreement on the particular norms ``when the consequences of generally following that norm for the satisfaction of the interests of each and every individual are acceptable to all by reasons of arguments'' 
%\label{ref:RNDWTecnMoUGX}(Brożek, 2007, p.172).
\parencite[][p.172]{brozek_rationality_2007}. %
 Following the detailed reasoning Brożek concludes that ``practical discursive rationality is an admissible interpretation of practical rationality as defined by Kant'' 
%\label{ref:RNDTs9weNSKio}(Brożek, 2007, p.173).
\parencite[][p.173]{brozek_rationality_2007}. %
 Although the reference to some form of universality expressed in the idealised discourse was supposed to solve the ``ideological vacuum'' of instrumental rationality, it has not done its job entirely. Discursive pragmatical rationality remains vouge and undecidable, at least within a~certain scope. Firstly it does not avoid the trap of the vicious circle. We see that to determine if the decision is rational, we need to determine the principles of rational discourse previously, and at least some of them are debatable. Should we really provide open access to the discourse for everybody who wishes? What about the anti-vaccine movement or global warming denialists? Should we introduce as a~principle the requirement of a~certain level of expert knowledge? Secondly, even the discourse which perfectly fulfils all the principles may lead to various solutions, equally possible and admissible though contradictory. Thus we find ourselves again in the point of departure, which is unresolved undecidability.

\section*{Rational choice theory}
Having sketched the three different concepts of rationality and pointing out their strong and weak points, let us turn again towards instrumental rationality, which is the base in contemporary mainstream economics. It is almost traditionally approved that the first philosopher who rejected the concept of metaphysical sources of good and evil, and thus the concept of metaphysical rationality was Thomas Hobbes. He reduced human moral actions to fulfilment of an individual appetite or desire, and therefore Brożek, instead of writing about instrumental rationality, writes of Hobbesian one. Actions subordinated to an individual desire aim to maximise individual utility, which recalls the utilitarian ethical doctrine and the definition of wealth in economics. Utilitarian ethics, in basic terms, seems primitive. Good is whatever is desired by an agent. But the shift in the mode of thinking was tremendous. Hobbes and their utilitarian disciples asked what an agent desires instead of pursuing her deemed metaphysical sources of good and evil or any universalities. Or, in other words, they watched her preferences revealed by her choices. For the sake of simplification, we will not go deeper into the problem of preferences themselves, i.e. whether they are revealed or hidden and whether they have different levels or not.\footnote{The problem of preferences is extensively considered in
%\label{ref:RNDnhdsl6AwEC}(Hausman and McPherson, 2006)
\parencite[][]{hausman_economic_2006} %
 and in 
%\label{ref:RNDDyMHUMhe17}(Kowalski and Kwarciński, 2016).
\parencite[][]{kowalski_racjonalnosc_2016}. %
 } The most crucial element referred directly to the topic of the paper is that, once we assume that the satisfaction of one's preferences is an end to be reached, and, due to the scarce resources, we cannot satisfy all of the preferences at any given time, we need rules to determine whether an applied order of preference satisfaction is rational or not. Those rules constitute rational choice theory which has three main sub-theories applicable in different circumstances. Those are ordinal utility theory, cardinal utility theory and game theory. All of them are pretty complicated when one goes into detail. Still, it is not the paper's subject to outline those details, but rather to catch the essential features, which would further be used to set them against the ``rationality'' principles empirically observed as applied by humans, which significantly decline from the patterns of ration choice theory.\footnote{More about rational choice theory can be found in 
%\label{ref:RNDUavbpvR1fi}(Hausman and McPherson, 2006; Kowalski and Kwarciński, 2016).
\parencites[][]{hausman_economic_2006}[][]{kowalski_racjonalnosc_2016}. %
 We wrote about it in 
%\label{ref:RNDwUhE6IBusJ}(Gorazda and Kwarciński, 2020).
\parencite[][]{gorazda_miedzy_2020}. %
 Game theory and its axioms are outlined in classical work of J.von Neuman and O.~Morgenstern 
%\label{ref:RNDiwof2iScVz}(1944).
\parencite*[][]{von_neumann_theory_1944}.%
}

The term utility has relatively negative connotations for philosophers and humanists and is usually bound to the monetary dimension. It is justified by the way how economists typically apply the term in their models. It is, however, a~simplification. Originally utility is connected with nothing more or less than preference satisfaction, regardless of the nature of preference. If an agent gives up a~crowded and noisy party with her friends and chooses the peaceful and silent prayer in the temple instead, it is a~sign of particular preference satisfaction, which has nothing in common with monetary values. In circumstances when an agent faces a~set of alternative choices, we may say that her revealed choices are rational if they meet two conditions, being axioms of ordinal utility theory, namely completeness and transitivity. The former requires that whenever an agent is set against alternatives to be chosen, she cannot withhold her choice or be irrelevant but should make a~decision on her preference, even if she is indifferent to alternatives being set before her. Thus completeness requires at least a~weak ordering of one's preferences. The latter axiom says that whenever an agent is faced with three alternatives A, B~and C, and she prefers A~over B~and B~over C, he must choose A~over C. If our preferences were intransitive, then we would risk being exploited by those with transitive preferences. Hausman and McPherson
%\label{ref:RNDjJ4n42c5Zs}(2006)
\parencite*[][]{hausman_economic_2006} %
 demonstrate the risk of exploitation by the so-called ``money pump argument,'' where each exchange according to one's preferences is connected to some monetary compensation expressing the difference in their strength. An agent with intransitive preferences, after several exchanges, remains with nothing. The proof is exciting and funny, but we also need to remember that this fundamental theory of rationality requires no valuation of our preferences. Under conditions of certainty, we do not need to determine how much we value particular goods we tend to choose or the exact difference between those we choose and those we reject. The case is radically different in a~situation of risk or uncertainty. Both are defined by the fact that an agent is incapable of determining precisely if the expected reward occurs after the choice is made. The risk is measurable, while the uncertainty is not. Measuring the risk means to ascribe to the alternative a~specific value (likelihood) from within 0 and 1, where 0 means absolute certainty that the expected reward will not occur and 1, the absolute certainty that the reward will occur. Whenever we have to make a~rational choice in such circumstances, we need to apply the cardinal utility theory, which means the minimum requirement to determine the measurable difference between alternatives. The difference needs to be measured as otherwise, we would not be able to calculate the ``expected value'' (or ``expected utility'') of a~chosen alternative. The latter is a~result of the multiplication of an estimated likelihood and utility value.\footnote{The axioms of expected utility theory is much more complex then ordinal utility. They were comprehensively formulated in 
%\label{ref:RNDKcnCRpfXe4}(Von Neumann and Morgenstern, 1944).
\parencite[][]{von_neumann_theory_1944}.%
} Once we do the proper calculations, we can compare the choices taking into account the measurable risk or uncertainty. Although very useful and elegant, the theory is full of paradoxes and traps, which make them in specific circumstances undecidable. Even worse, in daily choices, the people seem not to apply it or not to apply it correctly. The estimation of likelihood is the first barrier hard to overcome. D. Kahneman, the forerunner of research on human irrationality, in his book, \textit{Thinking Fast and Slow} 
%\label{ref:RNDtExZ2u7Q7K}(2011)
\parencite*[][]{kahneman_thinking_2011} %
 presents an example of such a~likelihood miscalculations the so-called ``Linda problem,'' well known also as a~conjunction fallacy. The participants of an experiment were supposed to read the basic characteristics of Linda. She, during her studies in philosophy, revealed concern on social justice and discrimination and political engagement in the anti-nuclear movement as well as superior intelligence. After having read the text, they were asked to guess her present occupation. Among the possible alternatives, there were also the ``bank teller'' and ``feminist bank teller,'' while most of the participants decided that it is more likely that Linda is a~``feminist bank teller'' than simply a~``bank teller.'' Such an assessment is in contradiction with the axioms of probability theory. The probability of the conjunction of two alternatives cannot be higher than the probability of each of those alternatives separately. The conjunction fallacy in probability assessment is quite common. There were also many attempts at the explanation why humans make this mistake systematically. One of the hypotheses presented by H.~Gintis 
%\label{ref:RNDbtrfEa0GFw}(2012)
\parencite*[][]{gintis_evolutionary_2012} %
 draws our attention to the applied narratives to which we are very sensitive. Once the information is included in the task description, we are strongly inclined to assume that the information is relevant to the solution. If Linda is presented as a~progressive activist, we automatically take that this feature will have a~strong impact on the deemed solution of the problem. Another good example of our inability to properly apply the expected utility theory is the famous Pascal wager. Blaise Pascal used the elements of decision theory for quasi-theological reasoning on the rationality of the religious stance. In his famous \textit{Pensées} 
%\label{ref:RNDQedeeeNfcN}(Pascal, 2003),
\parencite[][]{pascal_pensees_2003}, %
 he presented a~hypothetical wager, which can be formulated as a~decision matrix. One of the alternatives offers a~religious stance and a~chance for eternal life, while the other, a~godless life and the risk of eternal condemnation. In the matrix, the existence of God is a~determinant of a~possible redemption or condemnation and an uncertain event. Setting against the chance for infinite reward, even the tiny likelihood of God's existence makes the religious life a~better bet in the wager. While the reasoning presented by Pascal looks smart and flawless, in fact, it violates one of the axioms of the expected utility theory, the so-called monotonicity 
%\label{ref:RND5vtghFhDSv}(Jordan, 2006; Gorazda, 2009).
\parencites[][]{jordan_pascals_2006}[][]{gorazda_pragmatyzm_2009}. %
 The application of the infinite values in the decision matrix is not admissible because it makes the lotteries (alternatives) non-comparable. In two lotteries where the reward is of infinite value, it would not make a~difference in expected utility value regardless of whether the probability of winning is 1\% or 99\%. In both cases, the expected utility would be equal and infinite, too, while any rational player would doubtless choose the one with a~99\% probability of winning.

Pascal wager is also an example of the game theory, one of the most complex patterns of rationality. The game theory will have to be applied in the circumstances when in our expected utility calculation, we have to take into account another player or players and their respective choices. In the case of Pascal wager, the adversary player is nature itself, and the ``choice'' means the variable with two values---existence or non-existence of God. Those are typically other agents with their strategies in social relations, which must be considered before our decision is made. The complexity of game theory leads to paradoxes and mathematically proved multi-solutions or the undecidability of specific games. Moreover, in many experiments, it has been demonstrated that, even in simulations of games, which have undisputable solutions, men do not behave ``rationally.'' Two examples should be evoked; The so-called ``Prisoner's dilemma'' and ``Ultimatum game.'' The details of those games and experiments are widely described elsewhere.\footnote{Both games are comprehensively described in most of the handbooks for game theory e.g.
%\label{ref:RNDHmvuAG5uv0}(Straffin, 2002, 2004)
\parencites[][]{straffin_game_2002}[][]{straffin_teoria_2004}. %
 They are also outlined in 
%\label{ref:RNDL4ZYRPaS6c}(Hausman and McPherson, 2006).
\parencite[][]{hausman_economic_2006}. %
 } Needless to say, that in both examples, the players, participants in experiments, systematically choose the collaboration (in case of ``Prisoner's dilemma'') or costly punishment (in case of ‘Ultimatum game'') against the obvious, game-theoretical solutions; thus, they act irrationally according to this standard.

In respect to the primary subject matter of the paper, after the brief outline of the rational choice theory, two questions should be coped with. Firstly, the theory, which represents practical rationality (referred to the decision making), was constructed initially to be applied in the economic models. It represents economic rationality with its central assumption that rational decision is aimed at utility function maximisation. Can we use it in other than market areas of human life, where the decision are taken too? In other words, how much the theory can be considered universal. The positive answer is a~foundation of the so-called economic imperialism, i.e. the massive application of the economic assumptions and methods to beyond-the-market analysis. They were used among others in reference to law
%\label{ref:RNDmhJKI8ZZq2}(Stelmach, Brożek and Załuski, 2007; Posner, 1972);
\parencites[][]{stelmach_dziesiec_2007}[][]{posner_economic_1972}; %
 family affairs 
%\label{ref:RNDTeb3FDg3Mm}(Becker, 1981)
\parencite[][]{becker_treatise_1981} %
 and politics 
%\label{ref:RNDPtAIufOEeK}(Tullock and Buchanan, 1998).
\parencite[][]{tullock_calculus_1998}. %
 Whether those analyses were accurate, adequate and useful is another story that touches the problem of rational choice theory range of application which will be discussed in the second part of the paper. Secondly, there is a~problem of whether the theory is purely normative or positive. Does it exclusively represent the rules or instructions for rational decision-making, or does it also represents the actual behaviours of agents? The problem was posited by A.~Sen, who rightly noticed that ``The first and the most important use of rationality [...] must be normative: we want to think and act wisely and judiciously, rather than stupidly or impulsively. [...] Second, the use of ‘rational choice' in economics and related disciplines is very often indirect, particular as predicting device for actual behavior, and this can often overshadow the direct use of rationality. That indirect program is geared to the prognostication of actual behavior by first characterising rational behavior, and then assuming that actual behavior will coincide with rational behavior, or at least approximate it'' 
%\label{ref:RNDClwNPpg0az}(Sen, 2003, p.42)
\parencite[][p.42]{sen_rationality_2003}. %
 So, he ascribed to the rational choice theory two functions, normative and descriptive, giving precedence to the first and conditioned the second with the strong assumption of coinciding the actual behaviour with model rationality. The accuracy of this assumption is at least dubious and will be the subject of the next section.

Summing it up, however, we may say that the rational choice theory, with all its problems, paradoxes, multi-solutions or partial undecidability, is probably the most mathematically elegant and mostly decidable and helpful in the decision-making process. Giving primacy to individual preferences and withholding the judgment on their ``objective'' value is also devoid of double standards.

\section*{Ecological rationality, or what do people maximise}
The above-described concepts of rationality constitute the foundation of contemporary microeconomics and are components of the so-called \textit{Homo oeconomicus}. Economic man, which is supposed to maximise his utility function, is by default rational in terms of one of those theories, depending on the environmental circumstances. The problem with these foundational assumptions is that real men are not rational, either because many other variables determine their behaviour beyond the utility function or because man systematically violates the assumptions of rationality. Economists widely recognised the former problem from the very beginning
%\label{ref:RNDSRWLfh3tiR}(Mill, 2007).
\parencite[][]{hausman_definition_2007}. %
 They acknowledged that man was emotional, volatile in his preferences, irregular in his behaviour, and responsive to the particular social environment constituting the normative system. But, they also claimed that it should not change the condition rightly identified by Sen about the coinciding of the observed behaviour with the modelled rationality. Even if the patterns are not always followed, they are a~sufficiently good approximation of the actual aggregated behaviour of agents on the market and can be successfully used in economic models. First assaults against this mode of thinking came from two directions: From institutional economists, who rightly noticed the strong impact of socially constructed institutions on agents way of action 
%\label{ref:RNDoBJv2RqzdQ}(Veblen, 1994, 1898),
\parencite[][]{veblen_why_1898}, %
 and from the father of contemporary macroeconomics, J.M. Keynes. The latter claimed that market instabilities were often caused by a~specific feature of human nature, making us base our decisions on spontaneous optimism rather than mathematical calculations. The engine for our actions is our ``animal spirit'' 
%\label{ref:RNDuvik7jQ3yd}(Keynes, 2009).
\parencite[][]{keynes_general_2009}. %
 They were psychologists who drove the final nail to the coffin of economic man. Groundbreaking research was undergone and published by A. Tversky and D. Kahneman in 
%\label{ref:RNDWSb0auyR8g}(1979).
\parencite*[][]{kahneman_prospect_1979}. %
 After several experiments, they revealed a~set of agents' systematic declination from the rationality principles and the first significant cognitive biases:

\begin{enumerate}
\item Certainty effect, according to which agents overestimate results which are certain over those which are only likely;
\item Reflection effect, according to which agents are risk-averse if the risk is combined with the potential gain and risk-seeking if it is followed by potential loss.
\item Isolation effect, according to which agents overestimate the significance of the distinguishable elements of the alternatives and underestimate the elements which are common for them.
\end{enumerate}
This research was the beginning of numerous further experiments revealing subsequent cognitive biases. Apart from the above, they include ambiguity aversion, risk aversion, status quo tendency, framing effect, anchoring effect, mental accounting, endowment effect, sunk costs effect, hyperbolic discounting, probability matching and many others.\footnote{There are numerous papers and books where all those effects and experimental findings are widely described. Among others the recent book of D. Kahneman is worth mentioning
%\label{ref:RNDS5waG2ekWG}(2011),
\parencite*[][]{kahneman_thinking_2011}, %
 as well as handbook on experimental law and economics 
%\label{ref:RND8V8CxCUv65}(Arlen and Talley, 2008)
\parencite[][]{arlen_experimental_2008} %
 and more popular 
%\label{ref:RNDdHXteX5Gs2}(Petersdorff and Bernau, 2013; Shermer, 2008).
\parencites[][]{petersdorff_denkfehler_2013}[][]{shermer_mind_2008}.%
} Once those effects were discovered, an obvious question emerged: Are we chaotic in our choices, or is there another theory of decision that can be formulated consistently with the experimental findings? The early prospect theory and much later theory of two systems were attempts at such a~unification. However, the most promising area of research was evolution; as paraphrasing the famous evolutionary biologist, Theodosius Dobzhansky we may say that in social sciences (economics including) ``nothing makes sense except in the light of evolution'' 
%\label{ref:RNDR6GCudKLc3}(Dobzhansky, 1973)
\parencite[][]{dobzhansky_nothing_1973}.%


An evolution may give us a~clue to understanding the problem of rationality from a~different perspective. All of those normative theories of rationality and those which may be derived from prospect theory or two systems' theory are focused on individual utility. On the other hand, we know that agents are surprisingly often altruistic in their decisions, i.e. they seem to care more about the positive and negative utility of others than their own. They care about positive utility when they mean to increase the utility of some members of their reference group. They care about negative utility when they are eager to punish those delinquent or non-collaborative at their own expense. The above-quoted games like prisoners dilemma and ultimatum game are good examples. In experiments, subjects ``irrationally'' choose risky collaboration and costly punishment over the ``rational'' reward.\footnote{Those experiments and game matrixes in the background are explained in many handbooks for the game theory e.g.
%\label{ref:RNDA1Tp06zlAQ}(Straffin, 2002).
\parencite[][]{straffin_game_2002}. %
 Cultural differences in results are widely discussed in 
%\label{ref:RND8n3imyQXUM}(Henrich et al., 2001; Henrich, 2020).
\parencites[][]{henrich_search_2001}[][]{henrich_weirdest_2020}. %
 We write about them also in 
%\label{ref:RNDNMRHljVohf}(Gorazda and Kwarciński, 2020).
\parencite[][]{gorazda_miedzy_2020}.%
} There were many attempts at explaining humans' cooperation or altruistic behaviour, including those which tried to save the utility function and instrumental rationality by including into the agent's preference set those altruistic ones 
%\label{ref:RNDlibsVNULRv}(Becker, 1981).
\parencite[][]{becker_treatise_1981}. %
 Although compliant with the concept of revealed preferences, this trick does not answer the question of why we choose to prefer someone's good over our own. From the evolutionary perspective, whenever we observe the repeatable and relatively stable pattern of behaviour in the population, we may assume that it is or used to be adaptive for specific reasons. The assumption is corroborated by the fact that cooperative behaviours of agents put them at risk of being exploited by free-riders, other agents who do not intend to subordinate themselves to the collaborative order and pursuing their own interests, apply the instrumental rationality instead. If such a~risky behaviour evolved against all odds, it must have been adaptive even more. The most popular theories on the evolution of altruism include reciprocal altruism, kin altruism and group selection. The first one can still be considered within the individual utility paradigm according to the rule \textit{you scratch my back, I~scratch yours}. In other words, an agent acts altruistically because she counts on future reciprocity when she could benefit out of others' altruistic behaviour. It can work, but it does not explain all the actions in question. It has been proved through experiments that agents choose cooperative actions, even towards anonymous strangers, whom they have small chances to meet again. Kin altruism seems to be a~more comprehensive theory that found its mathematical representation in Hamilton's equation. The model assumes that our propensity towards altruism is a~function of genetic distance to the possible beneficiary of our actions; Closer relatives will be more likely to benefit than farther ones 
%\label{ref:RNDB7mgUGTLqF}(Hamilton, 1964).
\parencite[][]{hamilton_genetical_1964}. %
 Hamilton's proposal bases on the concept of a~``selfish gene,'' i.e. that in the biological evolution, those are genes and their proliferation which determines the value of fitness function and not the individual utility 
%\label{ref:RNDlygS5ZrVIZ}(Dawkins, 1976).
\parencite[][]{dawkins_selfish_1976}. %
 The shift in what is maximised here is remarkable, and it will be developed later. Group or multilevel selection is the most complex idea, which was firstly suggested by Ch.~Darwin as a~purely theoretical possibility and further developed mathematically by G. Price.\footnote{All those concepts of altruism are well outlined in the scientific biography of George Price 
%\label{ref:RNDC20nKXuXG3}(Harman, 2010).
\parencite[][]{harman_price_2010}.%
} Group selection assumes that in a~specific environment, an agent may bear the costs of her seemingly irrational behaviour, which at the same time benefit the whole group and brings about its sustainability. However, without those costly and unreasonable contributions, the group would not persist and, beyond the group, an agent's chances for survival and successful mating diminish. All of those theories lead to the more comprehensive theory of inclusive fitness, which encompasses the direct fitness (measured by a~proliferation of agent's genes) and indirect fitness (measured by a~proliferation of other's genes, while the kinship distance to those genes matters) 
%\label{ref:RNDjkh9DCmdWM}(Mouden et al., 2012).
\parencite[][]{mouden_what_2012}. %
 It is worth noting that, as soon as we put inclusive fitness in the first place, as the main subject of agent's maximisation, the individual utility usually composed of our preferences to pursue successful, convenient and happy life loses its primacy and appears to be a~useful social construct to manipulate our behaviour towards inclusive fitness. The evolutionary approach also solves the problem of hidden and revealed preferences partially. Although we may, to a~certain extent, decide according to our desires, we cannot freely choose what we desire.

If inclusive fitness is the leading determinant of our decisions and behaviour, the concept of practical, instrumental rationality as presumably ``coinciding with actual behaviour'' needs to be fundamentally reconstructed towards ecological rationality. There are at least a~couple of approaches to this rationality. Most of them emphasise the fitness to the given environment as a~significant driver of our behavioural patterns. To make the idea more familiar, we would focus on its forerunner and its well-known contemporary proponent. H. Simon
%\label{ref:RNDJkcKl67fsj}(1955)
\parencite*[][]{simon_behavioral_1955} %
 first noticed that our actual decisions surprisingly often do not coincide with model rationality. However, it does not mean that an agent is irrational or chaotic, but rather that she makes a~decision taking into account all the restraints in the access to the required information and analytical capacity to process them. Those are environmental restraints (information accessibility and complexity) and individual (available time and cognitive ability). Considering those ``transactional costs,'' the choices do not need to be optimal but only satisficing, and the rationality produced in such a~complex environment with restricted time and cognitive ability is bounded. The bounded rationality reveals itself in applying several identifiable heuristics, i.e. systematic modes of decision making, which are not compliant with the model rationality but allow users to reach satisficing results at minimum accessible information and cognitive engagement. Due to their simplicity, they occur to be especially useful in a~complex environment or situation when there is no time for deeper analysis. The theory was developed further by G. Gigerenzer, who proposed the definition of ecological rationality. It has been defined as ``the study of how cognitive strategies exploit the representation and structure of information in the environment to make reasonable judgments and decisions'' 
%\label{ref:RNDo0h2Va1DnX}(Gigerenzer, 2000)
\parencite[][]{gigerenzer_adaptive_2000}. %
 Consequently, the decision (or rather heuristic) is conceived as ecologically rational ``to the degree that is adapted to the structure of the environment'' 
%\label{ref:RNDUYTJTjv4WC}(Gigerenzer and Todd, 2012).
\parencite[][]{gigerenzer_ecological_2012}. %
 What counts here is not an ideal model of rationality but a~degree of adaptation measured by the inclusive fitness maximisation. The picture is much more nuanced. Firstly, ecological rationality is gradable, while classical rationality is, in principle, bivalent. A~decision is either rational or irrational, while it is more or less adaptive in the ecological approach. Secondly, ecological rationality loses its normative dimension. It is foremost descriptive. It studies agent's cognitive strategies and can merely be supplementarily applied to work out a~better (more adaptive) strategies for future problems. We can retrospectively assess the particular strategy as more or less adaptive or even maladaptive. Still, the assessment is valid exclusively in the given complex environment, which, even with hindsight, cannot be completely recognised.

Let us illustrate the differences between classical and ecological rationality using the allegory of an urn and coloured balls, which represents the worlds with different levels of uncertainty, decisions patterns, and consequences. In the first example, we draw white and black balls from the urn, having previous knowledge about the content of the urn. There are 100 balls, 25 black and 75 white. We do not need to be a~mathematician to bet on the white ball. In this world, anyone betting on a~black ball is clearly irrational. In the second world, we also have an urn with black and white balls, but we do not know how many of them are in the urn. We may watch the results of consecutive draws instead, and we know that after 40 draws, there were 10 black and 30 white balls selected. We apply the same calculus of expected utility, assessing the likelihood based on hitherto results. This ``inductive method'' is fallible, but the accessible alternative assumes pure uncertainty and indifference in betting on white and black balls. We should rather bet on the white ball based on our hitherto experience and strong assumption of world rationality.\footnote{The assumption of world rationality or unity of nature according to D. Hume is a~crucial element of justification of inductive method
%\label{ref:RNDjBWBSG2vwx}(Hume, 2000).
\parencite[][]{hume_treatise_2000}.%
} In the third world, the rules change. We still draw the balls for the urn and still watch the 25/75 distribution between black and white balls. However, we are now the member of the playing team. If any team member draws a~black ball, the rewards won by any other team members for white balls bet are forfeited. Respectively the rewards for an accurate bet on the black balls are forfeited if anyone draws a~white ball. The team which goes bankrupt (loses all wins after a~subsequent draw) cannot continue the game. My individual strategy should not change. The probability of drawing the white ball is still the highest, so I~should consequently bet on it. Even if I~lose my wins after someone has drawn a~black ball, I~will regain my rewards in the next lotteries, … unless my team goes bankrupt. If all the members will act instrumentally rationally, bankruptcy is inevitable. It can be mathematically proved that the optimal strategy requires 25\% ``irrational'' bets on the black ball.\footnote{The example is a~mathematical equivalent of a~story told by A.W. Lo 
%\label{ref:RNDiFerE3DN3P}(Lo, 2017)
\parencite*[][]{lo_adaptive_2017} %
 about Tribbles and their hypothetical settling strategy. He outlines mathematical proof of optimal strategy.} It is an example of ecological rationality. Suppose we replace the rewards with the number of offspring in this particular environment. In that case, the population playing exclusively instrumentally rationally will soon become extinct, leaving the ecological niche for the apparently irrational players.

\section*{Rationality in small and large worlds}
The fact that in a~given example, there is a~decidable and calculable optimal strategy seems to pour some optimism into pursuing the general theory of rationality, which could be normative and at least partially descriptive in Sen's terms. Knowing that some previously revealed irrationalities in human behaviour are in fact the expression of a~much wider concept of ecological rationality may indicate the possibility of the theory which can solve many problems at once, optimising the individual utility and sustainability of a~population. It may require some compromises, but in the end, in a~given environment, we might work out the optimal strategy. Access to such a~theory reminds the ``Darwinian demon,'' a~hypothetical creature who would be able to simultaneously maximise many different aspects of utility on different levels
%\label{ref:RNDNWC0J5o3cl}(Law, 1979).
\parencite[][]{law_optimal_1979}. %
 Unfortunately, the hopes are premature. Even if we forget about the ``naturalistic fallacy,'' posited by Hume 
%\label{ref:RNDjIGf0oaSjN}(2000)
\parencite*[][]{hume_treatise_2000} %
 and Moore 
%\label{ref:RNDce2ZuloBtr}(2004),
\parencite*[][]{moore_principia_2004}, %
 (which would simply express itself in the fundamental question: How do we know that ecological sustainability of a~population is good in principle?), the ``Darwinian demon'' seems to be an unachievable concept. The number of variables that would have to be included in the equations is too large and makes the undertaking incomputable, especially if we realise that the ``adaptiveness,'' being one of the ``utility'' to be maximised, is vogue and changeable in time. The adaptive effects of particular decisions vary in short-, medium- and long-term perspectives. Needless to say, that retrospective meta-analysis allows us to construct models which may explain only 2-5\% of the variation between natural populations 
%\label{ref:RNDhZYsYmYEsJ}(Mouden et al., 2012).
\parencite[][]{mouden_what_2012}. %
 It looks like we were not only irrational in classical terms but also unable to maximise inclusive fitness. Does it mean that we are condemned to decisional chaos? Not necessarily if we consider the comprehensive knowledge about the evolutionary processes. The fact that we are unable to explain a~variety of traits in terms of their adaptiveness is not so astonishing once we realise that the mechanism of evolution requires, in the first stage, a~``mutation,'' which is, by definition, random. In the cultural evolution, which shapes our decision patterns primarily, those ``mutations'' happen faster and oftener, among others, in the form of cultural drift, false imitations or innovative emulations (new solutions problems an agent faces). Without the environmental pressure, those traits may persist as they are adaptively irrelevant. The anthropological research suggests that the variability of the psychological characteristics in modern, western societies is much stronger than in primordial communities, which are surprisingly unified 
%\label{ref:RNDXb8ytJxIX6}(Henrich, 2020).
\parencite[][]{henrich_weirdest_2020}. %
 The same phenomena may be observed in reference to biological traits. Domestication of wild animals, which provides a~much safer environment, which does not demand the permanent fight for survival, leads to an astonishing biological variety 
%\label{ref:RNDwXHZJHhaqB}(Hare and Woods, 2020).
\parencite[][]{hare_survival_2020}. %
 The more a~random drift influences the investigated traits, the less evolutionary models can explain them. Moreover, natural selection acts upon the average consequences of particular traits. To make the population sustainable, it is not always necessary to reduce the ``irrational'' or somewhat maladaptive behaviours to null. It often suffices to reduce them to a~certain, evolutionary stable level. Natural selection also usually prefers ``cheap'' solutions over perfect ones. Therefore we watch the surprising success of decisional heuristics, which at first sight looks irrational but are found unexpectedly successful when costs of decision-making are considered 
%\label{ref:RNDhAI7C1S1AN}(Gigerenzer and Brighton, 2009).
\parencite[][]{gigerenzer_homo_2009}. %
 There are also many traits that evolved as a~by-product of selection for another gene.\footnote{Valverde et al. 
%\label{ref:RNDQJzMCq1nBY}(1995)
\parencite*[][]{valverde_variants_1995} %
 identify red hair as a~by-product of a~stronger ability to vitamin D~synthesis from UVB. Boyd and Richerson claims 
%\label{ref:RNDbqRLJgJxoT}(2005)
\parencite*[][]{richerson_not_2005} %
 that low fertility in modern societies is a~by-product of women education.}

Not everything is lost, however. Even in relatively safe environments, like modern western societies, there are still spots where the evolutionary pressure acts strong enough to shape the patterns of decision and make them eligible for a~relatively accurate description of agents' behaviour and a~robust normative pattern. The circumstances in which a~natural selection reveals its power can be described in four points
%\label{ref:RNDTLxs1JtjKy}(Mouden et al., 2012):
\parencite[][]{mouden_what_2012}:%


\begin{enumerate}
\item The decisions and their consequences can be precisely measured.
\item The choices are simple or routine so that there is no need to rely on heuristics, or
\item The stake is high so that the costs of decision-making are negligible in relation to the stake.
\item The individual is in total control of choice.
\end{enumerate}
The circumstances coincide, among others, with market behaviours. Though they are not resistible for some ``irrational'' biases, like hoarding or framing effect, the professional market players seem to be more deliberative in their investment decisions. They are, for instance, much less susceptible to the so-called endowment effect and much better is risk assessment
%\label{ref:RNDM5N6YVyeEL}(Arlen and Talley, 2008).
\parencite[][]{arlen_experimental_2008}. %
 Trained economists also seem to be more instrumentally rational in their decisions if they refer to the subject matter of their studies 
%\label{ref:RNDiKDU12iEU9}(Frank, Gilovich and Regan, 1993).
\parencite[][]{frank_does_1993}.%


In more general terms, we may say that the quoted elements constitute the so-called small worlds, where an agent has complete access to the decision matrix, defining the alternatives, their likelihood and consequences, and where the decision costs can be neglected. In contrast, we may define the large world, which can be characterised by its inherent uncertainty and knowledge deficit or high costs of their reduction in relation to the stake. If we refer to the example with urn and balls, we may specify the large world with the following features (alternatively or in combination)
%\label{ref:RNDPLu8wbrnRj}(Gigerenzer and Brighton, 2009):
\parencite[][]{gigerenzer_homo_2009}:%


\begin{enumerate}
\item Stochasticity. Balls drawn so far do not represent any pattern that could be a~base for further statistical inferences. Especially the variety and number of balls of the same colour are equal and randomly distributed, thus not allow to reduce the level of uncertainty.
\item Underspecification. It is the world in which I~face the possible lottery before the urn, but I~cannot make any reasonable bet, as I~have no previous knowledge about the content of the urn and no draws have been done so far, or the number of draws is highly insufficient to reason on the urn's content and distribution.
\item Misspecification. The world in which the assumed randomness generates confusing patterns. Although there were 100 draws in which white and black balls reveal a~60/40 ratio, the 40 black balls happened in the last 40 draws.
\end{enumerate}
The small and large worlds are not dichotomic. There is a~continuity between them, and rarely we can encounter the ideal types. Ideal small worlds are possible only in mathematics and in isolated settings, entirely designed by humans and non-inherently-interdependent (reflexive).\footnote{Inherent interdependency or reflexivity is often commented on and investigated feature of social sciences, where other agents responses to the exogenous variable in the model change the value of other exogenous variables in the model and thus makes it impossible to establish the value of the endogenous variable, or where there is an interdependency between exogenous and endogenous variables
%\label{ref:RNDaY4ZTKbZYi}(Soros, 2013; Beinhocker, 2013; Nowak-Posadzy, 2016).
\parencites[][]{soros_fallibility_2013}[][]{beinhocker_reflexivity_2013}[][]{nowak-posadzy_refleksyjnosc_2016}.%
} Artificially designed lotteries could be an example. Other worlds would represent large worlds, while some are smaller than others, like financial markets. The more the environment resembles a~small world, the more an application of the principles of classical rationality is justified and successful. Growing stochasticity, misspecification and underspecification pushes us towards more complex and sophisticated analysis and deployment of ecological rationality, which by definition is deprived of the normative nature and strongly depends on the particular environment.

\section*{Concluding remarks}
Classical rationality expressed the strongest in the rational choice theory seems to be the most precise and perfect decision theory, fitted to various contexts and environments, including decisions under certainty, risk, uncertainty and decisions taken against or in collaboration with other agents. It has a~solid normative component, but it widely fails to describe the behaviour coinciding with the theoretical predictions. Humans seem not to be rational in terms of that theory. Studying the systematic declines from classical rationality improves our understanding of agents' actual behavioural patterns, among others, by supplementing the theory with heuristics explained by ``transactional costs'' or by the concept of inclusive fitness and collective actions leading in the end to the ecological approach. The stronger we try to factualise our theory with empirical findings, the weaker remains its normative component, to be finally lost in the ecological rationality. On the other hand, those findings do not significantly improve our predictions but only make us aware of the subject matter's complexity. Moreover, the variety of all those models look like they being applicable in various situations and environments. Thus, applied and effective models of rationality seem to be situationally dependent and models are not discovered but rather designed by humans. Choosing the right model in the large world is highly challenging itself, and no metarules for such a~selection has been proposed so far. We are not, however, condemned to entirely chaotical actions. In the precisely defined circumstances where an agent can measure the decisions and their consequences, where the choices are simple or routine so that there is no need to rely on heuristics, or where the stake is high so that the costs of decision-making are negligible in relation to the stake and she is in total control of choice, the selected model of rationality can be effectively applied. The above circumstances constitute the small world. Beyond the small worlds, rule stochasticity, underspecification and misspecification and the only reasonable method are consecutive trials and errors, which eventually may reduce the large world to the small one. In other words, our optimal strategy is not to discover the best theory of rationality but to shrink the world.

\end{artengenv}
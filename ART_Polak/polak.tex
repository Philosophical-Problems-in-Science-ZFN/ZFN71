\begin{artengenv}{Paweł Polak}
	{Mathematics and metaphysics: The history of the Polish philosophy of mathematics from the Romantic era}
	{Mathematics and metaphysics: The history of the Polish philosophy\ldots}
	{Mathematics and metaphysics: The history of the Polish philosophy of mathematics from the Romantic era}
	{Pontifical University of John Paul II in Krakow}
	{The Polish philosophy of mathematics in the 19\textsuperscript{th} century is not a~well-researched topic. For this period, only five philosophers are usually mentioned, namely Jan Śniadecki (1756–1830), Józef Maria Hoene-Wroński (1776–1853), Henryk Struve (1840–1912), Samuel Dickstein (1851–1939), and Edward Stamm (1886–1940). This limited and incomplete perspective does not allow us to develop a~well-balanced picture of the Polish philosophy of mathematics and gauge its influence on 19\textsuperscript{th}- and 20\textsuperscript{th}-century Polish philosophy in general. To somewhat complete our picture of the history of the Polish philosophy of mathematics in those times, we here present the profiles of some lesser-known Polish Romantic philosophers of the 19\textsuperscript{th} century, namely Karol Libelt, Bronisław Trentowski, and Józef Kremer. We discuss their contributions to the philosophy of mathematics and their metaphysical perspectives, and we also show how their metaphysical ideas have found some continuity in the studies of some Catholic philosophers.}
	{history of Polish philosophy, philosophy of mathematics, Józef Hoene-Wroński, Karol Libelt, Bronisław Trentowski, Józef Kremer, Marian Morawski.}


\section*{Introduction}
\lettrine[loversize=0.13,lines=2,lraise=-0.01,nindent=0em,findent=0.2pt]%
{H}{}istories of the Polish philosophy of mathematics predominantly focus on the 1920s and 1930s, which was a~period of rapid development for the philosophy of mathematics, largely due to the developments and successes of Polish mathematicians
%\label{ref:RND9tzpmQjMAI}(e.g. Murawski, 2004, p.325).
\parencite[e.g.][p.325]{murawski_philosophical_2004}. %
 From earlier periods (i.e., pre-20\textsuperscript{th} century), however, few philosophers of mathematics are mentioned. This limited perspective derives from a~general lack of knowledge about the history of Polish scientific philosophy, with many earlier contributions being poorly acknowledged and remaining unrecognized.\footnote{A~very detailed historical background of Polish scientific philosophy in the 19\textsuperscript{th} century has been provided by Jan Woleński (2015), but he did not mention the 19\textsuperscript{th}-century Polish philosophy of mathematics, focusing instead on the interwar (1918–1939) period.} The most representative picture of this state for the history of the Polish philosophy of mathematics can be found in Murawski's book \textit{The Philosophy of Mathematics and Logic in the 1920s and 1930s in Poland}. 
%\label{ref:RNDxM8XUhTstm}(Murawski, 2014).
%\parencite[][]{murawski_philosophy_2014}. %
 In this, Murawski \parencite*[][p.1]{murawski_philosophy_2014} states:

\myquote{
In fact, before [the] 1920s and 1930s, no serious philosophical reflections on mathematics and logic existed in Polish science. Naturally, this does not mean that philosophical concepts concerning mathematics and logic, developed in interwar Poland […] were formulated in an intellectual vacuum and that earlier there had not been any reflections on mathematics and logic in Poland
%\label{ref:RNDheyFWgOB50}(Murawski, 2014, p.1).
%\parencite[][p.1]{murawski_philosophy_2014}.%
}
Murawski mentioned only ``six figures that exerted certain influences---each one made a~completely different impact---on the further development of the concepts in question''
%\label{ref:RNDC2vONqEsaA}(Murawski, 2014, p.1).
\parencite[][p.1]{murawski_philosophy_2014}. %
 These were Jan Śniadecki and Józef Maria Hoene-Wroński (turn of the 18\textsuperscript{th} and 19\textsuperscript{th} centuries) and Henryk Struve, Władysław Biegański, Samuel Dickstein, and Edward Stamm (turn of the 19\textsuperscript{th} and 20\textsuperscript{th} centuries). In a~footnote, Murawski also mentioned a~seventh person, namely Władysław Gosiewski (1900s).

Fresh research into the history of Polish philosophy has documented several interesting contributions to the philosophy of mathematics, with them surprisingly thought to have origins in Romantic (idealist) philosophy. This is surprising because the philosophy of the Romantic period is generally considered to be anti-scientific.

In this paper, we show how the reflection on mathematics in Polish philosophy developed over a~period ending in the early 1870s. We posit that the studies from this period created the foundations for philosophical reflection on mathematics in the 20\textsuperscript{th} century, despite the fact that the Polish philosophy of mathematics in the 20\textsuperscript{th} century broke with the previous century's ideas. In other words, the reflections on mathematics from the 19\textsuperscript{th} century were relatively quickly forgotten. The main aim of this paper is therefore to recall and analyze these forgotten, yet historically important, contributions to the Polish philosophy of mathematics.

In what follows, we analyze the development of the Polish philosophy of mathematics and reveal its unique and specific tradition of philosophical reflection.\footnote{It is interesting, from a~historical point of view, that even in a~time when positivism was supreme in Polish thought, the metaphysical approach to the philosophy of mathematics was still practiced. It may have been because of the tradition of scientific philosophy in Poland. This supposition was confirmed in the studies of the Krakow and Lwów schools for the philosophy of nature at the turn of the 19\textsuperscript{th} and 20\textsuperscript{th} centuries
%\label{ref:RNDRRubia9rJo}(Heller, 2019; Polak, 2019b, 2016; Heller and Mączka, 2007).
\parencites[][]{heller_how_2019}[][]{polak_philosophy_2016}[][]{polak_philosophy_2019}[][]{heller_krakowska_2007}. %
 In this paper, we question whether this tradition could also account for the specificity of the 19\textsuperscript{th}-century Polish philosophy of mathematics. } We begin by discussing the background to the Polish philosophy of mathematics in the 19\textsuperscript{th} century. Next, we present the precursors of this discipline and describe how Romantic (idealist) philosophy kindled an interest in the philosophical aspects of mathematics in later periods. Finally, we discuss how the metaphysical tradition in the philosophy of mathematics has influenced Catholic philosophy. The paper ends by giving some conclusions and general observations about the philosophy of mathematics' history in Poland.

\section*{Background to the 19\textsuperscript{th}-century Polish philosophy of mathematics}

Polish culture in the 19\textsuperscript{th} century was strongly influenced by a~very unfavorable geopolitical situation. Poland was partitioned between Russia, Prussia, and Austria (later Austria–Hungary). The loss of political independence and the subsequent persecution and suppression of the Polish language and culture strongly influenced 19\textsuperscript{th}-century Polish philosophy, and the philosophy of mathematics was no exception to this.

In the early 19\textsuperscript{th} century, the philosophy of mathematics was studied in four scientific centers: Wilno (now Vilnius in Lithuania), Poznań, Krakow, and Warszawa (Warsaw).\footnote{Hoene-Wroński, who worked mainly in France (especially Paris), was the exception here. Despite his works being written in French, Hoene-Wroński's ideas had a~strong impact on some Polish philosophers.} By the second half of the century, only Krakow and Warszawa continued these studies. At the turn of 19\textsuperscript{th} and 20\textsuperscript{th} centuries, a~new philosophical center appeared on the scene, namely Lwów, which is now Lviv in Ukraine.

\section*{Forerunners of the philosophy of mathematics in Poland}

One of the earliest Polish philosopher of mathematics\footnote{For example, one can mention here jesuit Adam Adamandy Kochański SI (1631--1700), a~mathematician and Polish philosopher of the early Enlightenment. He can be regarded as a~precursor of the philosophy of mathematics in Poland. In the light of current research it can be concluded that his correspondence with famous scholars, especially with Leibniz
%\label{ref:RNDqYcVXoMN7M}(see Kochański, 2005; Kochański and Leibniz, 2019; Polak, 2019a),
\parencites[see][]{kochanski_korespondencja_2005}[][]{kochanski_korespondencja_2019}[][]{polak_nauka_2019}, %
 contains many interesting remarks; however, apart from some metaphilosophical issues (mathematical philosophy), they do not have the fully crystallized character of philosophical reflection on mathematics. It is therefore worth undertaking a~systematic, in-depth study of this historically important issue.} was the mathematician, scientist, and philosopher Jan Śniadecki (1756-1830), who was one of the most interesting representatives of Polish Enlightenment philosophy 
%\label{ref:RNDUF6YHGmRxJ}(e.g. Murawski, 2014, pp.1–5; Straszewski, 1875).
\parencites[e.g.][pp.1–5]{murawski_philosophy_2014}[][]{straszewski_jan_1875}. %
 Śniadecki's Enlightenment ideas were not necessarily novel, yet they were distinguishable from other contemporary Enlightenment works because of their original arrangement 
%\label{ref:RNDTpEp5XLoWU}(Roskal, 1994).
\parencite[][]{roskal_jana_1994}. %
 Śniadecki's philosophy had some influence on philosophical studies at Vilnius University,\footnote{A~good example is Jacek Krusinski's paper ``On ways to usefully learn mathematical sciences'' [O sposobach pożytecznego uczenia się nauk matematycznych]. Krusinski discusses the nature of mathematics from the perspective of education 
%\label{ref:RNDj1QHvwpqVE}(Krusinski, 1806).
\parencite[][]{krusinski_o_1806}. %
 Another example can be found in Józef Twardowski's work entitled ``General remarks on the order of mathematical truths, especially Algebra and on ways of their interpretation'' [Ogólne uwagi nad porządkiem prawd matematycznych, szczególniej Algebry i~nad sposobami ich wykładania], which was published in \textit{Tygodnik Wileński} (1806). In his study of the history of mathematical and physical sciences in \textit{Wilno}, Józef Bieliński ironically mentioned that Twardowski's successor, Antoni Wyrwicz, also practiced the philosophy of mathematics. Bieliński accused Wyrwicz of teaching the philosophy of mathematics instead of mathematics itself. This is how Bieliński characterized those lectures: ``The general view of mathematics is that arithmetic is the study of the properties of numbers bounded by units; that lower algebra-magnitudes are not bounded by units; higher algebra that traces the properties of functions of manifold form; that in differential and integral calculus, the properties of all functions are expounded by increasing or decreasing their variable magnitudes; that the calculus of variations disassembles the properties of all functions in general by changing their form-[the lectures] though fair, somewhat but insufficiently understood and developed, aroused in him a~particular passion. [Ogólny pogląd na matematykę, że arytmetyka jest nauką własności liczb ograniczonych jednostkami; że algebra niższa---wielkości nie ograniczonych jednostkami: algebra wyższa, że śledzi własności funkcyj rozmaitej formy; że w~rachunkach różniczkowym i~całkowym wykładają się własności wszystkich funkcyj, powiększając albo zmniejszając ich zmienne wielkości; że rachunek waryacyjny rozbiera własności wszystkich w~ogóle funkcyj przez zmianę ich formy-aczkolwiek sprawiedliwe, poniekąd lecz niedostatecznie pojęte i~rozwinione, wzbudzało w~nim szczególne zamiłowanie]'' 
%\label{ref:RNDdAIzMpfqtP}(Bieliński, 1890, pp.21–22).
\mbox{\parencite[][pp.21–22]{bielinski_stan_1890}}.%
} as well as Krzemieniec High School, which depended on Vilnius University.\footnote{Interesting philosophical remarks about mathematics can be found in Wojciech Jarkowski's small book \textit{Mowa Woyciecha Jarkowskiego do uczniów przy rozpoczęciu kursu w~dniu 2. miesiąca października 1805 roku w~Krzemieniecu miana} [Woyciech Jarkowski's speech to the students at the beginning of the course on October 2, 1805 in Krzemieniec]. Wojciech Jarkowski (1767--1836), a~former student of Śniadecki, stated directly his master's influence 
%\label{ref:RNDA74GhlPja8}(Jarkowski, 1805, p.9).
\parencite[][p.9]{jarkowski_mowa_1805}. %
 In his view, mathematics is treated traditionally as an abstraction from reality, but in his interpretation, the mathematical nature of reality makes physics possible, which is interpreted as applied mathematics, so it is closer to the modern concept of mathematical nature. Jarkowski also opened an axiological reflection on mathematics by describing their ``virtues.'' This axiological approach was important in the 19\textsuperscript{th}-century debates around the role of mathematics in education, but this topic is beyond the scope of this paper. For the sake of completeness, it should be added that Wojciech Zborzewski's (1795--1860) attempts to construct a~philosophy of mathematics were also mentioned, but the manuscripts that were mentioned can no longer be found 
%\label{ref:RNDV25q7m3O8L}(Nowiny, 1845; Majorkiewicz, 1847, p.341).
\mbox{\parencites[][]{noauthor_nowiny_1845}[][p.341]{majorkiewicz_historya_1847}}. %
 } The Enlightenment tradition (including Śniadecki's philosophy) in the Polish philosophy of mathematics was relatively short-lived, however, with it having limited impact on other thinkers.

Many Polish philosophers in the 19\textsuperscript{th} century perceived Śniadecki to be the founder of Polish modern philosophy in general, yet his philosophy of mathematics did not gain much recognition. It is likely that Śniadecki's empiricism, as well as his cautious Enlightenment approach to the ontology of mathematics, was not inspiring enough for the subsequent generation of Polish philosophers.

A~major shift in Polish philosophy began over 1804--1817
%\label{ref:RNDEEYYXkV5nF}(i.e. with the rejection of the enlightenment's eclecticism see Tatarkiewicz, 1970; Jaworski, 1997).
\parencites[i.e. with the rejection of the enlightenment's eclecticism see][]{tatarkiewicz_jakiej_1970}[][]{jaworski_z_1997}. %
 This shift may be attributed to the works of Józef Maria Hoene-Wroński (1776--1853), the creator of an original idealist philosophy called ``Absolute Philosophy'' or ``Messianism.'' Hoene-Wroński was to become a~key figure in the 19\textsuperscript{th}-century Polish philosophy of mathematics.\footnote{During this period, we should mention Feliks Jaroński (1777--1827), who taught philosophy at the Department of Philosophy of the University of Krakow. In his book \textit{O~filozofii} [On philosophy], he analyzed the relationship between metaphysics and mathematics. According to him, metaphysics, geometry (i.e., mathematics), and physics all derive from philosophy in general. Jaroński considered algebra and geometry to be part of ontology and thus ``concerned with quantity'' 
%\label{ref:RNDBtTDn2mR8Y}(Jaroński, 1812, p.35).
\parencite[][p.35]{jaronski_o_1812}. %
 Jaroński also believed that metaphysics should strive for accuracy and clarity, such as in geometry and algebra: ``Mathematics supports philosophy by its method of proof and strict rigor in thought'' 
%\label{ref:RNDKcIcvTAMNW}(Jaroński, 1812, p.48).
\parencite[][p.48]{jaronski_o_1812}. %
 Through this textbook, Jaroński introduced Polish students to some elements of Kant's philosophy of mathematics (i.e., synthetic \textit{a~priori} cognition in mathematics) 
%\label{ref:RNDYaYHkAaXvH}(Jaroński, 1812, pp.156–157, 197).
\parencite[][pp.156–157,~197]{jaronski_o_1812}. %
 However, he did not take them directly from Kant's work but rather from a~book by Gottfried Immanuel Wenzel (1754--1809). Wit Jaworski 
%\label{ref:RNDarc0NA7jIe}(1997, p.52; similarly Ziemba, 1872, p.193)
\parencites*[][p.52]{jaworski_z_1997}[similarly][p.193]{ziemba_jan_1872} %
 was right to conclude that Jaroński's knowledge of Kant's philosophy was superficial, so he simply cannot be considered a~Kantist.}

Hoene-Wroński was without doubt the forerunner for the metaphysical perspective on mathematics in Polish philosophy. He was first to use the term ``philosophy of mathematics'' to denote the science of the principal laws of mathematics. He divided mathematics into \textit{algorithmie} (universal algorithms) and geometry. He caught the attention of Polish philosophers despite his major work being initially published in French
%\label{ref:RNDStBnOYPmbL}(Hoene-Wroński, 1811, see also his English work: 1820, p.14nn).
(\cite[][]{hoene-wronski_introduction_1811}, \cite*[see also his English work:][p.14nn]{hoene-wronski_address_1820}).%


Hoene-Wroński also introduced some key concepts into the Polish philosophy of mathematics. He conceptualized mathematics as a~science dealing with \textit{form} (based on distinguishing \textit{form} from the content of the physical world), with the forms of space and time being the ``true object of mathematics.''\footnote{Hoene-Wroński stressed that space and time are objects of mathematics when we consider them from an objective point of view as properties of the physical world
%\label{ref:RND7ozk1e6VGr}(see the footnote in Hoene-Wroński, 1811, p.2).
\parencite[see the footnote in][p.2]{hoene-wronski_introduction_1811}.%
} There is no doubt that Hoene-Wroński's philosophy developed under the influence of Kant's philosophy.\footnote{For the Kantian influence on Hoene-Wroński's philosophy, see e.g. 
%\label{ref:RNDCoelsdvt7a}(Wagner, 2014, chap.3, 2012).
\parencites[][]{wagner_wronskis_2014}[][chap.3]{wagner_wronskis_2016}.%
} He introduced the concept of the ``metaphysics of mathematics,'' which, according to him, consisted of objective laws for the objects of mathematics (i.e., the principal laws of mathematics).\footnote{See 
%\label{ref:RNDs88t8alR6p}(Wagner, 2016),
\parencite[][]{wagner_wronskis_2016}, %
 as well as 
%\label{ref:RNDwFrgN2f9PT}(Pragacz, 2007).
\parencite[][]{pragacz_notes_2007}. %
 Hoene-Wroński's original concepts were also briefly described by Roman Murawski 
%\label{ref:RNDraVIs5iTEp}(2014, pp.5–7),
\parencite*[][pp.5–7]{murawski_philosophy_2014}, %
 who pointed out that Hoene-Wroński's metaphysical attitude was typical for the Polish Messianist group, although this is not precisely characterized. For this analysis, it is more accurate to use the term ``Romantic philosophers,'' because their attitudes towards Messianism and Hegel's philosophy vary.}

Hoene-Wroński also considered the concept of infinity. He was aware that infinity is involved in the metaphysics of calculus, and he interpreted infinitesimal quantities as ``purely subjectives''
%\label{ref:RNDlJqD0c5hfh}(Hoene-Wroński, 1814, p.35nn).
\parencite[][p.35nn]{hoene-wronski_philosophie_1814}. %
 He stated that without the idea of infinity, mathematics would be impossible.\footnote{``Nous dirons plus: \textit{Ce n'est que par l'infini qu'est possible la science des Mathématiques}. En effet, sans l'infini, nous n'aurions, en Géométrie, que des lignes droites, et, an Algorithmie, que la simple sommation (addition et soustraction); et l'on voit bien si avec ces élémens grossiers, on aurait pu construire une science.'' 
%\label{ref:RNDzvelX2Wxp3}(Hoene-Wroński, 1814, pp.43–44})
\parencite[][pp.43–44]{hoene-wronski_philosophie_1814}} %
 His philosophical views led him to accept the idea of an actual infinity, a~concept that opened up many interesting philosophical questions.

Hoene-Wroński strongly influenced prominent Polish Romantic philosophers like August Cieszkowski, Bronisław Trentowski, and Karol Libelt
%\label{ref:RNDjEP4DKDn28}(e.g. Wójcik, 2013, p.16).
\parencite[e.g.][p.16]{wojcik_fenomen_2013}.%
\footnote{Traces of Hoene-Wroński's philosophy can also be found in the works of less famous philosophers of this era, such as Tyszyński 
%\label{ref:RNDnINNg6aHtZ}(1854, pp.81–82).
\parencite*[][pp.81–82]{tyszynski_rozbiory_1854}.%
} Hoene-Wroński is regarded as a~one of the guiding spirits of Polish mathematics in the late 19\textsuperscript{th} century 
%\label{ref:RNDFhTIsWirSI}(cf. Wójcik, 2013, p.15).
\parencite[cf.][p.15]{wojcik_fenomen_2013}. %
 While Hoene-Wroński's metaphysical tradition in the Polish philosophy of mathematics only lasted about a~century, some aspects of Wroński's philosophy are still of interest 
%\label{ref:RNDD69KRXCaf2}(Wagner, 2014, 2012).
\parencites[][]{wagner_wronskis_2014}[][]{wagner_2012}.%


\section*{Karol Libelt: Idealist origins of a~mathematical nature }

After Hoene-Wroński's studies, Polish philosophers of mathematics of the 1840s turned toward idealistic philosophy. The crucial philosophical problem then became about the relationships between mathematics and logic, as well as the ideas that grew out of the new Heglian logic, which was in reality a~kind of metaphysics.

This new direction of philosophy was characterized by rejecting the anti-metaphysical attitude of the Enlightenment. Very interestingly, and significant in this context, is Karol Libelt's critique of Jan Śniadecki's works. Libelt's study was published anonymously under the title ``Filologia, filozofia i~matematyka uważane jako zasadnicze umiejętności naukowego wychowania [Philology, philosophy, and mathematics conceived as fundamental domains of scientific education]'' in \textit{Tygodnik Literacki} 1838, no 17-20 and 28-29, and 1839, no 13-16
%\label{ref:RND0FMDLdzUPP}(Libelt, 1838, full edition 1850).
\parencite[][full edition 1850]{libelt_filologia_1838}.%
\footnote{Here for convenience, we quote the full edition 
%\label{ref:RNDnRQVefQRKu}(Libelt, 1850).
\parencite[][]{libelt_filologia_1850}.%
} Libelt had studied mathematics and philosophy in Berlin before teaching mathematics and publishing a~high school textbook 
%\label{ref:RNDplvPpfAJBY}(Libelt, 1844).
\parencite[][]{libelt_wyklad_1844}.%


Libelt
%\label{ref:RNDxZPTSu5urp}(1850, p.240nn)
\parencite*[][p.240nn]{libelt_filologia_1850} %
 ardently opposed Śniadecki's philosophical views, and he described his new philosophy by contrasting it with Śniadecki's ideas.

Mathematics, for Libelt, was a~discipline that was close to, and almost entwined with, philosophy. He found significant similarities between these two domains, such as deductive form and the style of abstract thinking. In Libel's view, both domains touched sensible and ``over-sensible'' objects through specific mind activity. Mathematics, Libelt posited, differed from philosophy in not being a~fundamental science, because mathematics needs some concepts, which it ``borrows from philosophy.'' He listed two basic, intuitive concepts, namely space and time, which were inspired by Kant's philosophy of mathematics as interpreted in the framework of Heglian philosophy
%\label{ref:RNDtEC3njxdbb}(Libelt, 1850, p.269).
\parencite[][p.269]{libelt_filologia_1850}.%


Libelt also drew a~distinction between pure and applied mathematics and highlighted the differences between the main domains of pure mathematics, namely geometry, algebra, and arithmetic.

In another article, Libelt
%\label{ref:RNDyEbOhMDnH3}(1842, pp.65–67)
\parencite*[][pp.65–67]{libelt_filozofia_1842} %
 presented an interesting view about mathematics' relationship to nature. According to him, reality embodies mathematical properties, so a~mathematician can consequently abstract these properties from reality:

\myquote{
Mathematical truths are purely objective, they are outside us, and in us only is the knowledge of these truths. Number and figure, like time and space, are forms abstracted from the world, full of latent content, or hidden properties, which mathematical reason has detected in them
%\label{ref:RNDtYiCC0f9a6}(Libelt, 1842, p.66).
\parencite[][p.66]{libelt_filozofia_1842}.%
\footnote{,,Prawdy matematyczne są czysto przedmiotowe, są zewnątrz nas, a~w nas tylko jest wiedza tych prawd. Liczba i~figura, jak czas i~przestrzeń są formy ze świata zdjęte, pełne utajonej treści, czyli ukrytych własności, które rozum matematyczny w~nich wykrywał.''}
}
This perspective led Libelt to formulate his controversial thesis: ``Mathematics does not teach anything about the abstract that does not exist in reality''
%\label{ref:RNDI3MEpxaJfq}(Libelt, 1842, p.65).
\parencite[][p.65]{libelt_filozofia_1842}. %
 That his view was so close to later concepts of the nature of the mathematical was made possible by his idealistic metaphysical interpretation. (This interpretation stated that if reality is founded as an ideal, it simply embodies ideal properties that, by abstraction, can be used in a~conceptual analysis.) Of course, Libelt's view differs from the modern concept of mathematics, because in his approach, mathematics is limited to objects abstracted from reality. In other words, Libelt rejected the possibility that pure (i.e., non-abstracted) mathematical objects exist. It is interesting that Libelt accepted the concept of infinity in philosophy yet believed that mathematics dealt only with finite values and sometimes only ``touches the infinity.''

Libelt also tried to characterize the relationship between mathematics and physical reality. He concluded that mathematical objects are approximations in an infinity of real, physical objects. In this view, mathematics is the oath between cognition \textit{a~priori} and \textit{a~posteriori}, a~kind of epistemic bridge between two domains of cognition. For him, mathematics dealt with the forms (of space and time), while metaphysics dealt with the contents,\footnote{,,Czym jest matematyka dla formy (przestrzeni i~czasu), tym jest metafizyka dla treści. Matematyka czysta tak w~geometrycznym, jak arytmetycznym swym dziale, jest równie oderwana, najściślej lo[g]iczna, wysnuwająca się z~najprostszych wyobrażeń przestrzennych i~liczbowych, w~zupełnej odrębności od materyalnego świata. A~przecież ciała niebieskie w~nieskończoności przestworza wirujące, słuchają tych prawideł elipsy, paraboli i~hiperboli, które matematyk abstrakcyjnie jako prawdy niewzruszone, w~tych liniach ostrokręgowych odkrywa; a~przecież prawa optyki i~mechaniki wedle prawideł matematyki czystej się tłomaczą. I~dla tego tak ogromnej wartości stała się matematyka zastosowana. Podobnie ma się z~filozofią spekulacyjną, co do treści otaczającego nas świata, co do znaczenia nas samych i~wszelakich ludzkości stosunków, co wszystko doprowadza nas ostatecznie do Boga, jako do ostatniej wszystkiego przyczyny''
%\label{ref:RNDXS4lAgIxnx}(Libelt, 1874, p.IX).
\parencite[][p.IX]{libelt_filozofia_1874}.%
} an observation that Libelt made in the introduction to the second edition of his book \textit{Philosophy and Critique}. His Platonic views were related to his idealistic philosophy 
%\label{ref:RNDYpz5zc1PnV}(Libelt, 1874, p.95),
\parencite[][p.95]{libelt_filozofia_1874}, %
 and they seem to have evolved from his earlier interpretations of mathematical objects 
%\label{ref:RNDBfqq6EWjsk}(Libelt, 1842).
\parencite[][]{libelt_filozofia_1842}.%


\section*{Bronisław Trentowski's post-Heglian approach}

In the 1840s, the most prominent Polish philosopher of mathematics was Bronisław Trentowski (1808–1869), who was the author of the influential work \textit{Chowanna czyli system pedagogiki narodowej}
%\label{ref:RNDinwg3jH1XX}(Trentowski, 1842a; Trentowski, 1842b).
\parencites[][]{trentowski_chowanna_1842}[][]{trentowski_chowanna_1842-1}.%


The relationship between mathematics and logic was especially described in this book
%\label{ref:RNDuVdxUFsVzV}(Trentowski, 1842b, sec.51).
\parencite[][sec.51]{trentowski_chowanna_1842-1}. %
 Later on, he deepened his analysis of the mentioned relationships in his next book entitled \textit{Myślini czyli całokształt lo[g]iki narodowej} 
%\label{ref:RNDSG9tn6oE6O}(Trentowski, 1844a; Trentowski, 1842b).
\parencites[][]{trentowski_myslini_1844-1}[][]{trentowski_chowanna_1842-1}. %
 Trentowski was working in Germany after migrating there, but he published his works mainly in Poznań.

Trentowski was the first Polish philosopher to use the Polish term for ``the philosophy of mathematics'' (i.e., \textit{filozofia matematyki}).\footnote{Hoene-Wroński actually used this term first, but as he wrote in French, the Polish translations of his books came after the publication of Trentowski's book.} For him, this branch of philosophy was part of the future philosophy of form (\textit{filozofia formy}). He described it as follows:

\myquote{
The philosophy of nature's form, which is space, time, and the shape of matter. This is geometry, arithmetic, and mechanics, so generally the philosophy of mathematics [Filozofia formy natury, którą jest przestrzeń, czas i~postać materyi. Jest to Jeometrya, Arytmetyka i~Mechanika, czyli w~ogóle filozofia matematyki]
%\label{ref:RNDkKe5bXGpJK}(Trentowski, 1844a, p.7).
\parencite[][p.7]{trentowski_myslini_1844-1}.%
}
The philosophy of form was conceived by Trentowski in opposition to the existing philosophy, which he denoted the ``philosophy of contents'' (\textit{filozofia treści}). For Trentowski, this distinction facilitated an analysis of the formal relationship between mathematics and logic. According to him, any similarities between mathematics and logic could only be formal, because they are both ``sciences of the form,'' but mathematics is based on forms of thinking rather than the abstraction of contents. Thus, he rejected any idea that logic could be reduced to mathematics and vice versa. In his view, mathematics and logic together constitute a~``science of form.''

Terntowki's other contributions to ideas in the philosophy of mathematics included a~reflection on the fundamental problems of modern philosophy, such as the ontological status of mathematical objects, the description of mathematics from a~metaphysical point of view as the ``science of external existence''
%\label{ref:RNDs3Lk9YNVAQ}(Trentowski, 1842b, p.324),
\parencite[][p.324]{trentowski_chowanna_1842-1}, %
 an attempt at explaining the psychological attitudes needed for mathematical thinking, and a~discussion of mathematical problems that were previously analyzed by Hoene-Wroński (e.g., infinity). With the benefit of hindsight, many of Trentowski's concepts seem very speculative, controversial, and sometimes even bizarre.

Trentowski's concept of mathematics as being a~branch of the ``science of form'' and the idea of a~mathematics–logic relationship were critiqued by Józef Ignacy Kraszewski (1812--1887). For Kraszewski
%\label{ref:RNDVqZjFvRud7}(1847, pp.28–29),
\parencites*[][pp.28–29]{kraszewski_system_1847}[][pp.28–29]{kraszewski_system_1862}, %
 Trentowski's concept of constancy for mathematics and logic was improvable. Some of Kraszewski's objections were hardly irrelevant, because for Kraszewski, mathematics was an abstraction from space, time, and shape, with it capturing only the quantitative aspects of physical reality.

\section*{Józef Kremer: Reflections on mathematics in a~Heglian philosophy framework}

A~special place in the history of 19\textsuperscript{th}-century Polish philosophy is occupied by a~philosopher from Krakow called Józef Kremer (1806–1875). Kremer was strongly influenced by Hegel, but he tried to build his own philosophy.\footnote{The most accurate description of the relationship between Kremer's and Hegel's philosophies can be found in
%\label{ref:RNDceQhD85e6L}(Struve, 1881).
\parencite[][]{struve_zycie_1881}.%
} In addition to Hegel's work, Kremer also studied the ideas of Trentowski and Libelt, and these influences can be specifically seen in the context of Kremer's pedagogic and aesthetics research 
%\label{ref:RNDoGyfM8aF5f}(Trentowski, 1842a; Trentowski, 1842b; Trentowski, 1844a; Trentowski, 1844b; Libelt, 1845).
\parencites[][]{trentowski_chowanna_1842}[][]{trentowski_chowanna_1842-1}[][]{trentowski_myslini_1844-1}[][]{trentowski_myslini_1844}[][]{libelt_filozofia_1845}.%


Kremer was chosen as an example here because he is the only well-known member of the Krakow philosophical milieu that was connected with the \textit{Towarzystwo Naukowe Krakowskie} (Krakow Scientific Society) in the 19\textsuperscript{th} century, so the philosophy of science in 19\textsuperscript{th}-century Krakow certainly warrants further study.

Like Trentowski, Kremer was interested in the relationship between mathematics and logic, but his concept of logic differed somewhat from that of his predecessors.\footnote{In his first philosophical publication
%\label{ref:RNDXUC53glhBU}(Kremer, 1835a; Kremer, 1835b),
\parencites[][]{kremer_rys_1835}[][]{kremer_rys_1835-1}, %
 Kremer made some cursory philosophical remarks about mathematics. Kremer's mathematical models were strictly limited in how they could be used to describe reality. Kremer thought that each level of organization in reality needed another set of concepts to describe it.} He described logic as a~``science, which has absolute thought and its development as a~subject'' 
%\label{ref:RNDisFbKgGL8L}(Kremer, 1849, p.112),
\parencite[][p.112]{kremer_wyklad_1849}, %
 such that it was a~discipline closer to the former metaphysics and ontology 
%\label{ref:RND7rolbzkPLo}(Kremer, 1849, p.114).
\parencite[][p.114]{kremer_wyklad_1849}. %
 In this view, mathematics became a~part of logic and ontology, with it aiming to describe the quantitative aspect of being. For Kremer, mathematics was a~part of general philosophy 
%\label{ref:RNDdTOHdwJeFa}(Kremer, 1849, p.9),
\parencite[][p.9]{kremer_wyklad_1849}, %
 but he perceived mathematics as applied general (i.e., theoretical) thought:

\myquote{
Mathematics is not the science of pure thought but rather the application of it to quantity and space, and as such, it is limited by quantity and space. Mathematics does not include philosophy, but philosophy includes mathematics.\footnote{,,Matematyka nie jest umiejętnością myśli czystej, ale raczej zastosowaniem jej do ilości i~przestrzeni: dla tego ogranicza się ilością i~przestrzenią; dla tego matematyka nie obejmuje w~sobie filozofii, ale filozofia zawiera w~sobie matematykę.''}
}
Kremer's studies were significant because they paved the way to think about logical and mathematical ontology, which is closer to the later, more modern, concepts of scientific philosophy.

\section*{Some remarks about the Polish philosophy of mathematics in the Romantic period}

The Romantic period, which was dominated by idealism, awoke interest in a~philosophical reflection on mathematics. The philosophers of this era began to perceive the deductive nature of philosophy as a~generalization of the deductive character of mathematics. The ideas of that period were strongly influenced by German philosophy, mainly Hegelian idealism, and any influence from other philosophical schools, such as the post-Kantian critical philosophy of the Friesian school
%\label{ref:RNDo9CPiXbXO9}(see e.g. Pulte, 2013)
\parencite[see e.g.][]{pulte_j_2013}%
\footnote{Post-Kantian philosophy was generally known to Polish philosophers to some extent. The history of philosophy by W.B. Tennemann, translated into Polish by J.H.S Rzesiński and published in Krakow, testifies to this 
%\label{ref:RNDuEuyCJfN3U}(Rzesiński, 1837, pp.184–189),
\parencite[][pp.184–189]{rzesinski_w_1837}, %
 as well as the encyclopedia entry 
%\label{ref:RNDorH6P7o214}(Fries (Jakób Fryderyk),
\parencite[][]{noauthor_fries_1862}. %
 We could even find a~reference to the Friesian method in Michał Wiszniewski's book 
%\label{ref:RNDf65g57alyy}(Wiszniewski, 1834, pp.169–170),
\parencite[][pp.169–170]{wiszniewski_bakona_1834}, %
 which was also published in Krakow, although it is interpreted as a~pre-positivist approach. In this book, Wiszniewski analyzed the methodological role of mathematics in science and revealed the limits of Bacon's concept of the inductive method 
%\label{ref:RNDqQyDQAnCdv}(Wiszniewski, 1834, pp.131–139).
\parencite[][pp.131–139]{wiszniewski_bakona_1834}. %
 Wiszniewski was a~graduate of the Krzemieniec High School, which had a~strong tradition of Śniadecki's philosophy. The philosophers associated with the Krzemieniec High School attempted to develop a~philosophy of mathematics in the spirit of the Enlightenment (see Jarkowski above).} is rather hard to locate.

One can find original ideas and even original concepts in the works of Polish philosophers (see the philosophy of Hoene-Wroński), but in general, reflections on mathematics had little to do with the practice of mathematics itself.

We could conclude that until the 1870s, the most important problems of the philosophy of mathematics for Polish Romantic philosophers, Wroński excepted, were the relationship between mathematics and logic and the ontology of mathematical objects. For most of these philosophers, mathematical objects existed as Platonic forms in an absolute mind. Unfortunately, most of those philosophers generally held a~very narrow concept of mathematics as a~formal science for quantitative aspects of reality. Because of this limited perspective, their reflection was unfruitful and uninfluential, and it is of merely historical significance now. Regardless of the rather limited import of these philosophers on the future development of philosophical ideas, their studies undoubtedly led to the philosophy of mathematics being later recognized as something important.

It is not easy to provide an objective evaluation of the contribution that this period made to philosophical thought. On the one hand, its studies demonstrated the importance of philosophical reflections on mathematics, but on the other hand, the use of too many oversimplifications hampered any development of new perspectives. Despite the lack of any lasting influence, this period of development for the Polish philosophy of mathematics introduced the metaphysical (ontological) interpretation of mathematics, which may well be its only lasting contribution.

Among the Romantic philosophers who studied the philosophy of mathematics were some lesser-known thinkers, but their contributions were regarded as being controversial and never gained wider recognition. For example, Józef Żochowski (1801–1851) mentioned the philosophy of mathematics as a~part of the philosophy of science in his book \textit{Filozofia serca, czyli mądrość praktyczna}
%\label{ref:RNDgiPUpSmbyu}(Żochowski, 1845, pp.1, 202–203).
\parencite[][pp.1, 202–203]{zochowski_filozofia_1845}. %
 However, his bizarre views on the relationship between mathematics and astronomy and his views on the nature of mathematics were curiosities rather than serious philosophical reflections, something that unfortunately fits well with the still-common stereotypes of romantic philosophy.

\section*{The philosophy of mathematics and modern Catholic philosophy: The case of Stefan Pawlicki and Marian Morawski }

The Romantic era created a~stable tradition of taking a~metaphysical approach to the philosophy of mathematics, although the level of these reflections and the degree of understanding for contemporary mathematics were very limited. This metaphysical tradition continued in a~specific, modernized Catholic philosophy, namely Neo-Scholasticism, which developed in the latter decades of the 19\textsuperscript{th} century. Given the differences between the Romantic philosophies and Neo-Scholasticism, it is important to note that these philosophical schools shared the common conviction that metaphysics must be defended from positivist anti-metaphysical attitudes.

The importance of Neo-Scholasticism grew rapidly from the 1870s, especially after the publication of the encyclical \textit{Aeterni Patris}
%\label{ref:RNDQltKLYTMkK}(Leo XIII, 1879).
\parencite[][]{leo_xiii_aeterni_1879}. %
 Catholic philosophy at that time was typically limited to modernized versions of medieval scholastic philosophies, and philosophers focused on scientific problems due to a~need to respond to the challenges posed by the positivist philosophical worldview. For the same reason, they ardently defended metaphysics and metaphysical thinking, and mathematics generally played a~minor role in scholastic philosophy due to Aristotle's methodological concept of \textit{metabasis}. In fact, the formal independence of the scholastic method of mathematics was the main obstacle in opening up a~dialogue with science, where explanations were essentially mathematical, even if the mathematical models were not explicitly formulated. In this context, it is worth exploring the possible influences that the Polish metaphysical tradition of mathematics had on Christian philosophy.

In Polish Catholic philosophy at the end of the 19\textsuperscript{th} century, two interesting reference cases for the philosophy of mathematics can be found in the works of two Krakow scholars, namely Marian Morawski (1845–1901) and Stefan Pawlicki (1839--1916). The discussions between these two philosophers, who represented two different schools of thought, gave shape to a~specific, localized (i.e., Krakow-centric) variety of Christian philosophy. Through the example of these two thinkers, we will show how the tradition of metaphysically reflecting on mathematics was adapted to the different approaches to create the modern Catholic philosophy.

\subsection{Marian Morawski: An apologetic use of the philosophy of mathematics }

Marian Morawski, a~Jesuit, represented the typical approach to modern Christian philosophy, which is regarded as a~revival of scholasticism. Although a~traditionalist, he practiced philosophy in an original way that had wide social impact. (His books were reprinted many times until the outbreak of World War II as an important aid to modern apologetics.). Morawski is regarded today as one of the most important figures in the early development of Polish Neo-Scholasticism.

In his most important monograph, \textit{Filozofia i~jej zadanie} [Philosophy and Its Task], Morawski
%\label{ref:RNDi8rlVvaU9d}(1877)
\parencite*[][]{morawski_filozofia_1877} %
 discussed the traditional concepts for the philosophy of mathematics, although this was unfortunately influenced by his apologetic perspective. Morawski treated mathematics as a~science of ``quantity and space,'' while philosophy provided its fundamental concepts and axioms 
%\label{ref:RNDhLckHtvVEA}(Morawski, 1877, pp.22, 27, 273).
\parencite[][pp.22,~27,~273]{morawski_filozofia_1877}. %
 Morawski thus interpreted the meanings of the axioms of mathematics realistically. This was typical for the Neo-Scholastic approach, but it also fitted well with the existing metaphysical tradition for mathematics. Morawski believed that the axioms of mathematics capture \textit{a~priori} the truth about the nature of formal entities 
%\label{ref:RNDqyy1Hndqqs}(Morawski, 1877, p.263 (footnote 1)
\parencite[][p.263 (footnote 1]{morawski_filozofia_1877}%
). Consequently, Morawski claimed that the axioms of Euclidean geometry are indisputable, because they express the truth about reality (Morawski, 1877, p.134). This was clearly an anachronistic position, since the Krakow Jesuit was evidently unaware of the existence of non-Euclidean geometries and their philosophical implications. What is worse, however, is that in subsequent editions of the book, this clear anachronism was not corrected 
%\label{ref:RNDw0EU4nzfCo}(see for example the third edition Morawski, 1899, p.163).
\parencite[see for example the third edition][p.163]{morawski_filozofia_1899}.%


Morawski's interest in the philosophy of mathematics was clearly related to his critique of August Comte's positivism
%\label{ref:RND0hr3N0VPaO}(Morawski, 1877, pp.174–175).
\parencite[][pp.174–175]{morawski_filozofia_1877}. %
 Morawski correctly criticized Comte's approach to mathematics by pointing out the purely deductive nature of the discipline, but this was for apologetic aims, so it could not effectively contribute to developing the philosophy of mathematics.

The apologetic use of some problems from the philosophy of mathematics can also be found in another widely propagated book of Morawski entitled \textit{Celowość w~naturze: studyum przyrodniczo-filozoficzne} [Purposiveness in nature: A~scientific and philosophical study]
%\label{ref:RNDDTOUxNCFOr}(Morawski, 1887).
\parencite[][]{morawski_celowosc_1887}. %
 In the first edition of this book, Morawski suggested that in their existence, natural entities are solving problems of a~mathematical nature 
%\label{ref:RND5GG4StcdaM}(Morawski, 1887, p.54).
\parencite[][p.54]{morawski_celowosc_1887}. %
 It is an interesting idea about the mathematical nature of reality, but it was only mentioned by Morawski incidentally.

He also explained the uniqueness of the concept of God by comparing it to a~mathematical concept, although God has a~genuine (i.e., non-abstract) existence. Since the second edition of this work, the specificity of mathematical concepts was stated directly: ``Only in the purely abstract sphere, as in mathematics, do concepts completely determine the essence of things---otherwise they are indefinite and useless, while our concepts of concrete things are alwaysincomplete and to a~certain extent indefinite''
%\label{ref:RNDsvEM3HrITj}(Morawski, 1891, pp.213–214).
\parencite[][pp.213–214]{morawski_celowosc_1891}.%
\footnote{,,Jedynie w~sferze czysto abstrakcyjnej, jak w~matematyce, pojęcia zupełnie określają istotę rzeczy --- inaczej są nieoznaczonemi i~nieużytecznemi, pojęcia zaś nasze o~rzeczach konkretnych są \textbf{zawsze} niezupełne i~w pewnej mierze nieokreślone [\ldots].''} It is not surprising that in this quoted fragment, Morawski recalled the classical concept of essence, demonstrating that the metaphysical tradition of the philosophy of mathematics was being directly adopted by the rising Neo-Scholasticism.

Morawski's approach became rather typical of later iterations of Neo-Scholasticism, so for more original ideas that developed within Catholic philosophy, we should look elsewhere.

\subsection{Stefan Pawlicki's original contributions}

An original contribution to the philosophical reflection on mathematics, in the context of modern Catholic philosophy, was offered by another priest, namely Stefan Pawlicki, who played a~very important role in reviving Polish Christian philosophy. He was not a~follower of Neo-Scholasticism, and his original philosophy can be partially explained by his unique background.

Stefan Pawlicki was an assistant professor at the Warsaw Main School during its short existence from 1862 to 1869, and there he taught philosophy with Henryk Struve. Their philosophy, while linked closely with science, was ``positivistic in spirit''
%\label{ref:RNDKQbvCXCap3}(Jadacki, 1997, p.147)
\parencite[][p.147]{jadacki_warsaw:_1997} %
 while still retaining their own original, critical, point of view. In 1868, Pawlicki went to Rome and became a~priest before turning to Christian philosophy. He became the first chair of Christian Philosophy at the Faculty of Theology of Jagiellonian University in Krakow (1882). He later moved to the Faculty of Philosophy at the same university (1894). His nomination was connected with the movement to restore Christian philosophy that was announced by the Encyclical \textit{Aeterni Patris} of Pope Leo XIII 
%\label{ref:RNDO7cEw2nzY1}(1879).
\parencite*[][]{leo_xiii_aeterni_1879}. %
 Pawlicki was directly nominated by the Pope for his position. (For more on this topic, see the works of Głombik 
%\label{ref:RNDndO9258nwQ}(1973)
\parencite*[][]{glombik_czlowiek_1973} %
 and Kępa 
%\label{ref:RNDtIrB2g2aIq}(2000)
\parencite*[][]{kepa_ks._2000}%
). This gave Pawlicki the freedom to pursue his philosophical search and allowed him to formulate an exceptionally original conception, at least for the time, of Christian philosophy.

Pawlicki was a~famous historian of philosophy, but he also created a~new Christian philosophy that differed from the typical Neo-Scholastic one. This philosophy was based on both science and personal experience (i.e., self-knowledge). Through its ideas, Pawlicki influenced the development of the philosophy of science in Krakow
%\label{ref:RNDNSeIWCvZnr}(Polak, 2017).
\parencite[][]{polak_rola_2017}. %
 We can find interesting remarks about the philosophy of mathematics in his main book \textit{Kilka uwag o~podstawie i~granicach filozofii} [Some Remarks on the Foundations and Limits of Philosophy] 
%\label{ref:RNDk6Afm77XC7}(Pawlicki, 1878).
\parencite[][]{pawlicki_kilka_1878}.%


In Pawlicki's view, the entirety of science is tied up with philosophy, with the borders between science and philosophy being fuzzy, permeable, and historically changeable. Mathematics was the first science that evolved from philosophy, so these domains of knowledge must be genetically close. For Pawlicki, every science needed philosophy to provide proof for its fundamentals, based on Aquinas' sentence: ``\textit{nulla scientia probat principia sua, sed probat alia ex eis} (\textit{Summa theol}. p. 1 q. t~a. 8)''
%\label{ref:RNDU3h86lXq5r}(Pawlicki, 1878, p.7).
\parencite[][p.7]{pawlicki_kilka_1878}.%


Pawlicki strongly critiqued the 19\textsuperscript{th}-century idealism. However, from the Polish idealists, he and Struve accepted the idea of mathematics ``borrowing'' some concepts from philosophy. This led him to stress the importance of philosophy to mathematics in justifying its foundations, something that could only be done by metaphysical philosophy
%\label{ref:RNDDHdkfd9f2q}(Pawlicki, 1903, p.459 (footnote 2)
\parencite[][p.459 (footnote 2]{pawlicki_historya_1903}%
).

In the footnotes and comments to his \textit{History of Greek Philosophy}, Pawlicki also made some interesting remarks about mathematics. In his critique of Plato, he was inspired by Struve's concept of meta-mathematics. He also quoted Dickstein's concept of mathematics and reality relationships. He also acknowledged the traditional concept of mathematics as a~formal science. He used all these concepts to critique Plato's concept of real existence for mathematical objects like triangles
%\label{ref:RNDA7pXMANLvV}(Pawlicki, 1903, pp.469–470).
\parencite[][pp.469–470]{pawlicki_historya_1903}. %
 He further posited that the relationship between mathematics and reality had been accurately described by Dickstein.

Pawlicki's view on the philosophy of mathematics was not original, but he successfully used Struve's and Dickstein's concepts in his critical evaluation of ancient philosophy. This demonstrates the openness and novelty of Pawlicki's philosophy, a~feature that is striking when compared with the works of other Christian philosophers, including beyond Polish Neo-Scholasticism.

In summary, the use of the contemporary philosophy of mathematics was an important innovation in the history of philosophy at the turn of 19\textsuperscript{th} and 20\textsuperscript{th} centuries.\footnote{For more information on Pawlicki's methodology for historiography and philosophy, see Mróz
%\label{ref:RND78ODo58BVZ}(2008).
\parencite*[][]{mroz_metoda_2008}.%
} Indeed, it represented the most important and original contribution to the development of the Polish tradition of reflection on mathematics that had been made by Christian philosophy in the 19\textsuperscript{th} century.

\section*{Conclusions: Metaphysics for mathematics as a~distinctive feature of the 19\textsuperscript{th}-century Polish philosophy of mathematics}

The purpose of this study was to further our academic understanding of the history of the Polish philosophy of mathematics. We have reviewed the relatively short history of the Enlightenment's reflection on mathematics that was connected with the figure of Jan Śniadecki, something that developed mainly in Wilno and Krzemieniec. We have shown that the mainstream philosophical thought, which was later called the metaphysical tradition, was formed from metaphysical considerations about mathematics that connected with a~clearly defined concept of mathematics as a~science for the quantitative aspects of reality. This tradition, which began with Hoene-Wroński's absolute philosophy, continued developing until the 1870s.

We observed that in this tradition, which was originally introduced by Hoene-Wroński and inspired by the Kantian philosophy of mathematics, there was an almost constant collection of concepts and problems. Surprisingly, the Polish Romantic philosophy inspired by Hegel, which is generally badly regarded in the philosophy of science, turned out to be fruitful for the development of the Polish philosophy of mathematics. The crucial contribution that was made here was the new concept of logic that Hegel introduced, because this inspired a~re-examination of the relationships between mathematics and logic.

Further research could explore how this stable tradition evolved into the modern Polish philosophy of mathematics of the 20\textsuperscript{th} century. In other words, it is worth analyzing the process by which the metaphysical approach came to be rejected in the philosophy of mathematics in Poland.

\end{artengenv}

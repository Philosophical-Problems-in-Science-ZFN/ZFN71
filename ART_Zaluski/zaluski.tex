\begin{artengenv}{Wojciech Załuski}
	{Conceptions of paternity and evolutionary psychology}
	{Conceptions of paternity and evolutionary psychology}
	{Conceptions of paternity and evolutionary psychology}
	{Jagiellonian University}
	{Evolutionary psychology offers a~fairly ‘patriarchal' picture of sex differences, according to which men are, ‘by nature', (a) much more polygamously disposed, (b) much more desirous of power over the opposite sex (this desire manifests itself in their more intense sexual jealousy), and (c) much more aggressive than women. However, the picture---at least in its components (a) and (b)---becomes problematic if one looks at the history of conceptions of paternity accepted by our ancestors. It is argued in the paper that the very fact that our ancestors accepted various and essentially different conceptions of paternity casts a~shadow of doubt on the ‘patriarchal' picture of sex differences (especially if this fact is coupled with the hypothesis that our \textit{most distant}-Pleistocene-ancestors accepted the conceptions which deny or marginalize the role of father in the process of the generation of children).}
	{evolutionary theory, patriarchal ethos, conception of paternity, ignorance of paternity, partible paternity.}




\section{Introduction}
\lettrine[loversize=0.13,lines=2,lraise=-0.01,nindent=0em,findent=0.2pt]%
{T}{}he paper's goal is twofold: to provide a~survey of the conceptions of paternity assumed throughout human history and to trace the implications of the fact that these conceptions were numerous and essentially different for the discussion about differences between male and female nature. More precisely: regarding the second goal it will be argued that the fact that human beings throughout their history adopted various conceptions of paternity (from those which imply that men play at least as important role in the generation of children as women do, to those which either deny the role of the father in the process of children's generation or minimize his role) may be relevant for the evaluation of the picture of sex differences offered by evolutionary psychology (especially if this fact is coupled with some hypotheses regarding the conceptions of paternity accepted by our most distant ancestors). The analysis conducted in this paper is divided into three parts. Section \ref{zal:sec2} presents four main conceptions of paternity that were accepted throughout human history. Section \ref{zal:sec3} deals with the problem of which of those conceptions was accepted by our earliest-Pleistocene-ancestors. Section \ref{zal:sec4} invokes the the conclusions from previous sections in the discussion about the differences between male and female nature.

\section{The multiplicity of the conceptions of paternity}\label{zal:sec2}
If one delves into the history of humankind, one will notice that our ancestors tended to endorse at least four different conceptions of paternity. According to \textit{Conception 1}, which I~shall call the conception of ‘\textit{paternal and maternal duo-genesis}', a~child cannot come into being without the participation of mother and father, and in the process of the child's generation neither of them plays a~more important role than the other---the ‘contribution' of each of them is equally necessary (and jointly sufficient) for the generation of a~child. This is the conception which we accept contemporarily, and which seems obvious to us (as it is supported by scientific knowledge). But it was by no means obvious for our ancestors. Some of them did accept it, but this acceptance was by no means universal. Three other conceptions were also endorsed. One of them is \textit{Conception 2}---that of ‘\textit{paternal monogenesis}'\footnote{I~borrow the name of this conception from Delaney
%\label{ref:RNDDAgBoCDPHM}(1986).
\parencite*[][]{delaney_meaning_1986}.%
}---which assumes that in the process of generation of children the ‘contribution' of father is much more important than that of mother: \textit{the former is essential for the identity of a~child}. Thus, even though both mother and father participate in ‘giving rise' to a~child, only father \textit{generates/begets} the child; mother could be substituted because she performs the inferior role of just being of ‘vessel' for father's semen; any other woman could replace her in this role \textit{without influencing the child's identity}. This means that, in the process of child's generation, father is active and \textit{creative}, whereas mother is passive; the only role (subsidiary, as one could say) she plays in this process is to give him/her nurture and birth. Thus, according to this conception, ``the male is said to plant the seed and the woman is said to be like the field'' 
%\label{ref:RNDmJ9X6aaezz}(Delaney, 1986, p.495).
\parencite[][p.495]{delaney_meaning_1986}. %
 One interesting implication of this picture is that sexual infidelity of a~woman is particularly harmful for her husband: the child is above all a~\textit{man's} child, so if the husband were to invest his resources in rearing not his own child, he would invest not (as in \textit{Conception 1}) in the child of \textit{his wife} and \textit{another man} but \textit{simply in the child of another man}, so this child would be \textit{entirely alien} to him. Consequently, it may not be accidental that the conception of paternal monogenesis was part and parcel of extremely patriarchal ideologies (e.g., the Muslim), where the control of female sexuality is particularly strong (through early marriage, veiling, seclusion, to infibulations and clitoridectomy). According to \textit{Conception 3}---that of \textit{the ignorance of physiological paternity}---the father plays no physiological role in child generation: there are no ties of blood between the child and the father (the latter being only a~sociological concept). One could query why this conception is not called that of ‘maternal monogenesis'. The answer is simple: in those societies where it is believed to have been accepted, e.g., among the Australian Aborigenes or the tribes from the Trobriand Islands, investigated by Bronisław Malinowski 
%\label{ref:RNDwxqiec7I6w}(Malinowski, 1913, 1927b; 1927a),
\parencites*[][]{malinowski_family_1913}[][]{malinowski_father_1927}[][]{malinowski_sex_1927}, %
 mother is not considered to generate a~child \textit{by herself} but with the (crucial) participation of ancestral spirits (\textit{Baloma}).\footnote{I~shall omit the conception of maternal monogenesis in my analyses, since there are no reasons to believe that it was ever assumed, i.e., that it was ever believed that mother---by herself---can generate a~child.} The conception of the ignorance of physiological paternity implies that sexual intercourse is not a~necessary condition of pregnancy; it is only \textit{helpful}, as it is one of the ways of opening the mother's birth canal (breaking the hymen) for letting the ancestral spirit in (the Trobriand Islanders believed that sexual intercourse could be replaced in this role, e.g., by digital manipulation). Accidentally, it can be remarked that even though, on this conception, the role of sexual intercourse in the generation of children was downplayed, the Trobriand women in fact pursued a~very active sexual life, also before marriage. \textit{Conception 4}---let me call it, after Sarah Blaffer Hrdy (2001), that of ‘\textit{partible paternity}'---assumes that in the process of generation the ‘contributions' of mother and father are equally important, but one father is not sufficient for the generation of a~child: for this generation to take place, more than one man must inseminate the child's mother.

\section{Primitive humanity and its conception of paternity }\label{zal:sec3}
In order to examine in versatile fashion the implications of the multiplicity of the conceptions of paternity for the picture of human nature, one would wish to know one more fact, viz. what conception of paternity was assumed by our most distant ancestors.Not surprisingly, this question is immensely difficult to answer, given the paucity and the selective character of the available pre-historical data: we can say with a~relatively high degree of certainty in which historical societies or types of societies a~given conception was endorsed but we can only speculate as to which of them was assumed (more or less consciously) by our most-distant-Pleistocene-ancestors.

\textit{Conception 2} (paternal monogenesis) is characteristic for societies with an advanced patriarchal culture, in which the role of women is marginalized. One could also argue that this conceptions is somehow connected with monotheism: ``the doctrine of monotheism is the fullest expression---the apotheosis---of the folk monogenetic theory of procreation''
%\label{ref:RNDHGmDwNRaji}(Delaney, 1986, p.502).
\parencite[][p.502]{delaney_meaning_1986}. %
 But it needs to be stressed that the connection between this conception and monotheism is by no means necessary. This conception was also assumed, e.g., in the polytheistic ancient Greece. Its fullest theoretical expression can be found in Aristotle's treatise \textit{On the Generation of Animals}, in which he maintained that a~child' essence (form) is given by the father's semen---mother gives only the body (matter) to the child.\footnote{One can also add that Aristotle was strongly patriarchal not only because he believed that women play an inferior role in procreation; he additionally believed that female nature is deficient in itself because woman's (deliberative) soul (unlike man's) does not have full mastery over body (the physiological basis of this difference was, in Aristotle's view, the fact that female soul is generated when semen does not have the sufficient heat).} The first philosopher who abandoned this conception was---in the 2th century---Galen, who endorsed the duo-genetic conception (\textit{conception 1}): he posited the existence of female sperm. Now, given that \textit{conception 2} seems to be confined to the later stages of the evolution of human societies, it would be rather implausible to maintain that it was accepted by our Pleistocene ancestors. For as it was convincingly argued by Barbara Smuts 
%\label{ref:RNDtqEFDuaUCX}(1995),
\parencite*[][]{smuts_evolutionary_1995}, %
 a~large number of conditions had to be satisfied for an advanced patriarchal culture to develop. The conditions she points at are the following: ``a reduction in female allies, elaboration of male-male alliances, increased male control over resources, increased hierarchy formation among men, female strategies that reinforce male control over females, the evolution of language and its power to create ideology'' 
%\label{ref:RNDfG3niv4k1g}(Smuts, 1995, p.20).
\parencite[][p.20]{smuts_evolutionary_1995}. %
 In her view, the third and the fourth conditions (``increased male control over resources'' and ``increased hierarchy formation among men'') could not be satisfied in primitive societies of our most distant ancestors. The male control over resources could not appear in foraging and nomadic societies: it could appear only ``with the advent of intensive agriculture and animal husbandry [...], when women's labor is restricted to a~relatively small plot of land, as in intensive agriculture, or is restricted primarily to the household compound, as in animal husbandry, it is easier to control both the resource-base upon which women depend for subsistence and women's daily movements'' 
%\label{ref:RNDJEcfSPKP76}(Smuts, 1995, p.16).
\parencite[][p.16]{smuts_evolutionary_1995}. %
 As for the ``increased hierarchy formation among men'', Smuts writes insightfully that ``the degree to which men dominate women and control their sexuality is inextricably intertwined with the degree to which some men dominate others'' 
%\label{ref:RNDWeXDU58E6u}(Smuts, 1995, p.18).
\parencite[][p.18]{smuts_evolutionary_1995}. %
 Thus, patriarchal ideology (whose expression was \textit{Conception 2}), could develop only if large inequalities between men had arisen beforehand. One may also notice that \textit{Conception 2} seems to require a~large dose of abstract thinking: indeed, it is by no means easy to come up with a~conception that totally negates the role of woman in the process of child generation.

By contrast, \textit{Conception 1} is less ideological and more commonsensical that \textit{Conception 2}; it also seems to require a~lower level of abstract thinking. But one must be careful in the evaluation of this conception. There is no doubt that it is commonsensical (and obvious) \textit{for us} (who know, from the science of genetics, that the contributions of father and mother to the child's genetic make-up are equal). But it may not have been so obvious for our ancestors. As already mentioned, Malinowski convincingly argued that the Australian Aborigenes and some tribes in the Trobriand Islands ignored physiological paternity and found the ‘obvious' \textit{Conception 1} (preached to them by the Christian missionaries) ridiculous. If one calmly considers the arguments that the Trobrianders invoked for their own conception of the ignorance of paternity (e.g., that those women who lead a~sexually active life do not necessarily become pregnant, that ‘ugly' women who are avoided---or rather \textit{usually} avoided---by men nevertheless become pregnant), one may concede that, from a~truly commonsensical or naïve, point of view, the connection between sexual intercourse and pregnancy must seem tenuous. Accordingly, one may speculate that it might have seemed tenuous also to our most distant ancestors. This hypothesis was endorsed by Malinowski himself: he claimed that \textit{Conception 3} was assumed not only by the Australian Aborigenes and Trobrianders but also by ``primitive mankind'': ``This ignorance is of general sociological importance, because there are well-founded reasons for believing that it was once universal amongst primitive mankind, as may be held to be proved by Mr. E.S. Hartland in his thorough treatise on \textit{Primitive Paternity} [...] primitive humanity was certainly wholly ignorant of the process of procreation''
%\label{ref:RNDMnV4sDdrhg}(Malinowski, 1913, p.181,200).
\parencite[][p.181,200]{malinowski_family_1913}. %
 Malinowski invoked also some other scientific authorities to support his claim about the ignorance of paternity among the primitive humans, e.g., James George Frazer and Arnold van Gennep. The claim was also endorsed, though in a~different form, e.g., by Johann Jakob Bachofen 
%\label{ref:RNDX9bAStgrKt}(1967),
\parencite*[][]{bachofen_myth_1967}, %
 developing a~hypothesis of matriarchy (``hetaerism'') as a~first phase in human history, Lewis Henry Morgan 
%\label{ref:RNDq1jf2jrF2C}(1870)
\parencite*[][]{morgan_systems_1870} %
 and Friedrich Engels 
%\label{ref:RNDM1SrNYWY1p}(Engels, 2004).
\parencite*[][]{engels_origin_2004}. %
 The ignorance of paternity which Bachofen, Morgan and Engels meant when they wrote about primitive societies, was not grounded in a~\textit{conception} of paternity but was simply a~certain fact resulting from (in their view) the historically earliest form of sexual relationship, viz. ``promiscuity'', that is: ``group marriage''---``the form in which whole groups of men and whole groups of women belong to one another, and which leaves but little scope for jealousy'' 
%\label{ref:RNDHaRbUGx3cf}(Engels, 2004, p.50).
\parencite[][p.50]{engels_origin_2004}. %
 But, needless to say, neither Malinowski's particular hypothesis that the Australian Aborigenes and Trobrianders ignored physiological paternity, nor---especially---his more general hypothesis that also ``primitive humanity'' ignored it, were unanimously accepted by anthropologists.\footnote{It was rejected, e.g.. by the Finnish sociologist Edward Westermarck; for a~detailed overview of the discussion around Malinowski's hypotheses see, e.g. 
%\label{ref:RNDRNnQLi6A0C}(Pulman, 2004).
\parencite[][]{pulman_malinowski_2004}.%
} Given that the anthropologists have not reached a~consensus on these complicated issues, it would not be prudent to profess any \textit{strong} opinion on these them. But if I~were to express \textit{some tentative} opinion, it would be along the following lines. Malinowski's particular hypothesis appears to be convincing. What seems to me to be a~particularly strong argument for it is a~rather puzzling (at first sight) combination of customs in the Trobriand society, viz. of the very loose sexual morality for unmarried women and of the condemnation of their having children. This combination becomes logical only if one assumes that the Trobriand Islanders ignored (physiological) paternity, i.e., that they did not clearly see the connection between sexual intercourse and pregnancy. But it is indeed an open question whether also ``primitive humanity'' ignored paternity.

What about \textit{Conception 4}? Can it be plausibly argued that it was adopted by our most distant ancestors? The case for the affirmative answer to this question becomes strong if we accept Sarah Blaffer Hrdy's description of the primitive societies. The core of her description is the claim that in those societies men were not ``reliable providers or protectors'', i.e., their parental investment was low. And, as she argued, ``wherever fathers prove unreliable providers or protectors, it makes sense for mothers---if they are free to do so---to line up one or several ‘secondary fathers'{''}
%\label{ref:RNDhaFLG52q1C}(Hrdy, 1999, p.xxiii).
\parencite[][p.xxiii]{hrdy_woman_1999}. %
 In this view, the main reasons why in primitive societies ``providers'' could have been ``unreliable'' was that ``in societies like the Aché adult as well as child mortality rates are high. Fathers may die; others may sire their children and then defect. Under some economic circumstances, it just may not be feasible for one man to provide for a~family'' 
%\label{ref:RNDnUh2tLpGJI}(Hrdy, 1999, p.xxiii).
\parencite[][p.xxiii]{hrdy_woman_1999}. %
 This is description of the Aché societies (an indigenous tribe of Paraguay)\footnote{One should add that the presence of the conception of partible paternity was ascertained in ``at least eighteen South American cultures, widely separated geographically [...] Indigenous cultures in India and Polynesia also accept partible paternity'' 
%\label{ref:RNDlOFe24HnEc}(Hubin, 2003, p.70).
\parencite[][p.70]{hubin_daddy_2003}.%
} but it may suit, in Hrdy's view, also the more primitive societies, in which our genotype was being shaped. She argued that, in those societies, women must have developed certain strategies to cope with the fact that men are ``unreliable providers''. One of their (possible) strategies was the strategy of associating with multiple males (sequentially or simultaneously) in order to ensure resources from them and to obtain their protection from other males. This last protection was important because, as Hrdy claimed, ``what mothers and infants most urgently needed a~male for was to protect them---not just from predators but from conspecific males'' 
%\label{ref:RNDZzDsp3UNbF}(Hrdy, 2009, p.148).
\parencite[][p.148]{hrdy_mothers_2009}. %
 Another important female strategy to deal with the ``unreliable providers'' was, in Hrdy's, view, the strategy of cooperation with other women in upbringing their progeny: the role of a~community in rearing children was therefore essential: ``a Pleistocene mother responsive enough to make her baby feel secure was likely to be a~mother embedded in a~network of supportive social relationships. Without such support, few mothers, and even fewer infants, were likely to survive'' 
%\label{ref:RNDlRKMMYeyUH}(Hrdy, 2001, p.101).
\parencite[][p.101]{hrdy_past_2001}. %
 It should be added that this ``cooperative breeding'' or ``allomaternal care'', as Hrdy calls this phenomenon, which made women less dependent on men, was not only a~way of dealing with the fact that men were rather unreliable providers; it also created, as Hrdy 
%\label{ref:RNDbxBrFIoFWr}(2009)
\parencite*[][]{hrdy_mothers_2009} %
 argued, a~selective pressure for developing more general (i.e., manifesting themselves not only in the context of breeding) cooperative and altruistic traits. The picture of female nature presented by Hrdy opposes the view that females' continuous sexual receptivity and concealed ovulation, as well as their capacity to experience multiple orgasms,\footnote{As Hrdy put it: ``The paradoxes of human sexuality---the mismatch between men, who are transiently impotent after an orgasm, and women, who are not only capable of multiple orgasms but may prefer them---may not be so paradoxical after all, if we no longer assume that these traits evolved in a~strictly monogamous context. The physiology of the clitoris, which does not typically generate orgasm after a~single copulation, ceases to be mysterious if we put aside the idea that women's sexuality evolved in order to ‘serve' her mate, and examine instead the possibility that it evolved in order to increase the reproductive success of primate mothers through enhanced survival of their offspring'' 
%\label{ref:RNDkTP4BdyfWT}(Hrdy, 1999, p.176).
\parencite[][p.176]{hrdy_woman_1999}.%
} serve the monogamous purpose of cementing the pair-bond; they may rather contribute to confusing paternity, and thereby to inducing more men to invest in a~woman's child (or at least to dissuading them from harming the child). This picture (contrary to the ‘patriarchal' one) implies that women are sexually assertive, and the female intra-sexual competition and variance in reproductive success is much greater than assumed or implied by evolutionary psychology, since women compete for the attention of multiple males 
%\label{ref:RNDaBA7poIStc}(cf. Hrdy, 1999, p.132).
\parencite[cf.][p.132]{hrdy_woman_1999}. %
 Now, if primitive societies were such as described by Hrdy, ``secondary'' fathers may have been indeed extremely useful, even indispensable for the survival of the offspring, and \textit{Conception 4} may have played an important role in legitimizing the multi-male strategy of women; as Hrdy put it:

\myquote{
A~useful biological fiction that fetuses are built up by semen from several men, widespread in this part of South America, facilitates maternal strategizing. Data for the Aché of Paraguay and the Bari of Venezueal indicate that children who receive gifts or food from several ``possible'' fathers have significantly higher survival rates than those receiving food from only one
%\label{ref:RNDH0qeLDgEzK}(Hrdy, 1999, p.xxii).
\parencite[][p.xxii]{hrdy_woman_1999}.%
}
She wrote in similar vein in her paper \textit{The Past, Present, and the Future of Human Family}:

\myquote{
I~like to imagine that it was a~cagey white-haired grandmother who first invented---thousands of years ago---the folktale to beat all folktales in terms of its helpfulness to her daughters. According to this folk mythology---which by now has spread over a~vast area of South America, encompassing peoples belonging to six different language groups---each foetus has to be built up from installments of semen contributed by all the men that a~woman has had sex with in the ten months or so prior to birth. Although in fact women do not bear litters sirred by several fathers they way wolves, jackals, and other cooperative breeders do and there is no such thing as a~human baby with more than one genetic father, this biological fiction about partible paternity has proved extremely convenient for mothers who needed to elicit extra assistance rearing their young and getting their children fed [...]. The South American belief in partible paternity facilitates cooperative provisioning
%\label{ref:RNDCd0GLgOfgj}(Hrdy, 2001, pp.92---93).
\parencite[][pp.92--93]{hrdy_past_2001}.%
}
Two remarks regarding the above picture of primitive societies should be made here. First, one could object to this picture by pointing out that in human societies polyandry is rare. But Hrdy aptly notices that ``across human cultures, polyandrous marriage in the formal sense is indeed extremely rare. But \textit{informally} [Hrdy's emphasis---W.Z.] everything from wife-sharing and sequential fathers to surreptitious adultery is far from rare''
%\label{ref:RNDTkdiUZYXPb}(Hrdy, 1999, pp.xxii---xxiii).
\parencite[][pp.xxii--xxiii]{hrdy_woman_1999}. %
 Secondly, it is not entirely clear what is the direction of causation in this picture: whether \textit{Conception 4} is the cause of women's seeking ``secondary fathers'' and the fathers' acceptance of women's multi-male strategy, or (as Hrdy seems to believe) it is a~fiction (and known---at least by women---to be a~fiction) invented by women to justify their multi-male sexual strategy, which was especially useful in environmental/economic conditions in which they lived (clearly, the latter interpretation adds another dimension to female nature: that of being crafty and manipulative, capable of outsmarting men, who appear on this picture to be somewhat slow-thinking and naïve). This point is obscure and cannot be easily clarified.

Let me summarize. It is clear that due to the paucity or even inaccessibility of the necessary data one cannot definitely resolve the question of what conception of paternity was accepted by ``primitive humanity''. As Carol Delany aptly remarked: ``despite numerous ethnographic studies and lengthy discussion, the question has still not been satisfactorily answered''
%\label{ref:RNDXuIalYTGtb}(Delaney, 1986, p.495).
\parencite[][p.495]{delaney_meaning_1986}. %
 But what also follows from this declaration of intellectual humility is that it is arbitrary to assume that ``primitive humanity'' accepted \textit{Conception 1} or \textit{2}; it is equally or even more plausible to maintain that ``primitive humanity'' was ignorant of (physiological) paternity, and thereby accepted \textit{Conception 3}; \textit{Conception 4}---that of partible paternity---is somewhat complicated, and for that reason arguably less likely than \textit{Conception 3} to have been accepted by our ancestors (though, of course, it cannot be excluded that it was in fact accepted).

\section{Implications for the discussion about sex differences}\label{zal:sec4}
In my analysis of the implications of the above conclusions for the problem of sex differences I~shall assume that ``primitive humanity'' assumed \textit{Conception 3} or \textit{Conception 4}; this is a~hypothetical assumption but, as argued above, there seem to be quite good reasons to treat it (especially as regards \textit{Conception 3}) as quite plausible.

Let me start from briefly presenting the picture of sex differences offered by evolutionary psychology. According to this picture men are: (a) much more dissolute (promiscuous), and thereby much less sexually discriminating than women; (b) much more desirous of power (\textit{as such} and \textit{over the opposite sex}) than women; (c) and much more aggressive than women (all these differences between sexes are assumed to be \textit{substantial}---hence the phrase ‘much more'). The argument for this picture consists of two parts.\footnote{A~more loosely formulated version of this argument can be found, e.g., in
%\label{ref:RNDDoucsxh0Mz}(Ridley, 1994; Buss, 1998).
\parencites[][]{ridley_red_1994}[][]{crawford_psychology_1998}. %
 For a~more general and synthetic overview of the types of reasoning conducted by evolutionary biologists see, e.g. 
%\label{ref:RND1Zh4tI5U54}(Kozłowski, 2018).
\parencite[][]{kozlowski_ewolucja_2018}.%
} \textit{Its first part} is a~theory of parental investment 
%\label{ref:RNDC0R0pqPEX1}(cf. Trivers, 1972),
\parencite[cf.][]{trivers_parental_1972}, %
 according to which the members of the sex (male sex in the human and most other species) whose parental investment in the offspring is lower and thereby whose maximum (potential) number of offspring is higher will be less discriminating in choosing the sexual partner, more polygamously inclined, and more intensely competitive (aggressive) with other members of the same sex. This part is supposed to explain why men are much more dissolute and much more aggressive than women (by this kind of behaviour men, unlike women, can increase the likelihood and extent of their reproductive success), but it does not explain why men are considered to be much more desirous of power (\textit{as such} and \textit{over the opposite sex}). This last feature (the desire of power over the opposite sex) is explained by the \textit{second part} of the argument, viz. the two-element assumption that: (a) in the ancestral environments (in which human genotype was being shaped) parental investment of men in their offspring was relatively large (mainly because of the prolonged helplessness of human infants) as compared with the other species (though, of course, still smaller than that of women), and (b) that a~man cannot be certain of his paternity of the child whom he rears. In other words, so the argument goes, since men invested relatively much in their offspring, but could not be certain of their paternity, they are likely to have developed an adaptation that diminished the risk of their misplaced (that is: in a~child of another man) parental investment. This adaptation can be most generally described as men's will of power over their sexual partners, or as male sexual proprietariness, or as men's tendency to treat their sexual partners ``as a~chattel'' 
%\label{ref:RNDEbazFfyNuR}(cf. Wilson and Daly, 1992).
\parencite[cf.][]{wilson_man_1992}. %
 It consists of various more specific ‘mental mechanisms', the most important of them being, as is claimed by evolutionary psychologists, is intense sexual jealousy.\footnote{It should be noticed that some evolutionary psychologists present a~more complex and sophisticated picture of sex differences. For instance, David Buss and David Schmitt 
%\label{ref:RNDxKt9qTXlXa}(cf. Buss, 1998; Schmitt, 2015; Buss and Schmitt, 2011)
\parencites[cf.][]{crawford_psychology_1998}[][]{schmitt_fundamentals_2015}[][]{buss_evolutionary_2011} %
 developed a~‘sexual strategies theory', arguing that both men and women have evolved a~complex repertoire of sexual strategies---a~‘pluralistic mating strategy': on the one end of the continuum there is long-term mating (extended courtship, the emotion of love, large investment of resources), on the other end there is short-term mating (casual sex, one-night stands, fleeting sexual encounters), and between these two ends: brief affairs, prolonged romances 
%\label{ref:RND2QlQNfQfV8}(cf. Schmitt, 2015, pp.207---271).
\parencite[cf.][pp.207--271]{schmitt_fundamentals_2015}. %
 The choice of a~strategy (or their mix) will depend on such factors as, for instance, opportunity, personal mate value, sex ratio, cultural norms, or parental influences. But even though the sexual strategies theory departs from the patriarchal picture of the differences between men and women by claiming that both men and women may have good (though different) evolutionary reasons for engaging in short-term mating, it still retains its important element: men are still regarded as more dissolute and in long-term relationships their desire of control over the opposite is regarded as stronger than that of women.}

Now, one can plausibly argue that if natural selection endowed men and women with proclivities which the patriarchal picture ascribes to them, it would be fairly improbable (though perhaps not impossible) that they develop and endorse \textit{at the earliest stages of human history} such conceptions of paternity (\textit{Conception 3} or \textit{Conception 4}) which are evidently contradictory to this picture. In point of fact, the mere variety of the conceptions of paternity, some of them being contradictory to the patriarchal picture of sex differences, is a~problem for those who believe that ‘patriarchal' sex differences were produced by natural selection (obviously, evolutionary psychologists do not deny the fact that human nature is flexible, but if this flexibility oversteps a~certain borderline, this constitutes a~challenge to their theory; of course, the question, which I~do not propose to answer, is whether the mere variety of the conceptions of paternity can already be regarded as overstepping this borderline). But the problem becomes especially acute if these non-patriarchal conceptions were adopted by our earliest ancestors. For our earliest ancestors can be expected to have manifested human nature (as it was shaped by natural selection) in its ‘purest' form. So if they accepted one of these conceptions and acted on them, one could hardly assert that patriarchal picture aptly describes ‘natural' (evolved) differences between men and women.

If ``primitive humanity'' accepted \textit{Conception 3}, and thereby, if men themselves believed that they cannot be (physiological) fathers, they could not have been concerned with the problem of the ‘uncertainty of paternity'; the problem simply did not exist for them. Consequently, if they did not regard themselves as (physiological) fathers of their children, their ‘parental investment' would have been arguably smaller than assumed by this argument. It does not necessarily mean that they would not have invested at all in their children, since they could have done so as a~‘side-effect' of their sexual attachment to the children's mothers. But this investment would have been arguably smaller than assumed in the patriarchal picture. As a~result, they would have displayed much weaker sexual jealousy than implied by this picture. Their jealousy would above all serve the protection of their access to their sexual partners, not the elimination of the risk of misplaced (not in their own children) parental investment (and as such could have been experienced with equal strength also by women): a~man cannot make any conscious efforts to increase his certainty of paternity if he does not know the very concept of (biological) paternity.

So far I~have traced the implications of the assumption that our ancestors accepted \textit{Conception 3} only from male perspective. What implications, however, does this conception have for the picture of our ``primitive'' female ancestors? Was she, as implied by this picture, ‘coy', \textit{univira}? Clearly not. If a~woman can be less fearful of male jealousy, and is less guarded by men than implied by the picture, then her sexual license is less risky. On this view, our female ancestors enjoyed much freedom, and the ideal of female chastity and strict sexual morality (with respect to women) appeared much later---with the onset of Neolithic revolution; as Bertrand Russell wrote: ``the discovery of fatherhood led to the subjection of women as the only means of securing their virtue---a subjection first physical and then mental, which reached its height in the Victorian age''
%\label{ref:RNDlPemOuq1mU}(Russell, 1929, p.12).
\parencite[][p.12]{russell_marriage_1929}.%
\footnote{One may note that even patriarchal institutions can be seen as an argument for the non-patriarchal view of women: ``There can be no doubt from such evidence [concerning the various ways in which men strove to control female sexuality---W.Z.] that the \textit{expectation} of female ``promiscuity'' has had a~profound effect on human cultural institutions'' 
%\label{ref:RNDoxSq6Hqgnw}(Hrdy, 1999, pp.176---177).
\parencite[][pp.176--177]{hrdy_woman_1999}.%
}

The assumption that \textit{Conception 4} was adopted by our most distant ancestors is even more difficult to reconcile with the patriarchal picture of sex differences proposed by evolutionary psychology; in fact it is blatantly contradictory to it. First of all, the acceptance of \textit{Conception 4} would have made men behave in a~strikingly ‘un-patriarchal' way: it would in fact motivate a~man who wishes to have a~child to encourage his female sexual partner to mate with other men. Furthermore, the very idea of rearing a~child of another man acquires a~rather peculiar sense if one accepts \textit{Conception 4}, and thereby believes that more than one man must inseminate a~woman for a~child to be conceived: on this conception, by rearing one's own child, one \textit{ipso facto} rears the child of other man or men. The implications of \textit{Conception 4} for understanding female nature were already mentioned in section 3; as was argued there, women do not prove to be ‘by nature' passive or chaste---in short: the coy female is a~myth
%\label{ref:RNDDZ4ostXw7P}(cf. Hrdy, 1986).
\parencite[cf.][]{hrdy_empathy_1986}. %
 Female nature is multi-layered and complex: it embraces in itself the capacity to be passive, coy, and chaste, but also the capacity to be active, resolute, even libertine.\footnote{A~similar picture emerges, e.g., from the excellent paper on ‘female infidelity and sperm competition' written by Shackelford, Goetz, Pound, and Lamunyon 
%\label{ref:RNDEAgvIpAGc2}(2015).
\parencite*[][]{shackelford_female_2015}.%
} Women are able to cooperate with other females if it is necessary, but also to compete with them. They may also become more aggressive than implied by the patriarchal picture: as mentioned, within-sexual competition between women is likely to be high especially if it is assumed (as in Hrdy's, account) that women compete for multiple males.

In summary, the very fact that in the history of mankind various conceptions of paternity (including those which cannot be reconciled with the patriarchal picture) were assumed casts a~shadow of doubt on the claim that natural selection produced sex differences implied by their ‘patriarchal picture'. This argument gains additional strength if it is true that the non-patriarchal conceptions of paternity were assumed by our earliest ancestors. However, the positive consequences of this argument are less clear: the argument assuredly undermines the ‘patriarchal picture' but it does not have clear implications as to what other picture is the proper one. Two different pictures come out as alternatives: the picture which retains, though in a~much reduced form, some elements of the patriarchal view (women as ‘by nature' sexually more restrained, more coy, and less aggressive than men) and the picture which presents a~radically different view (women as ‘by nature' sexually liberated, assertive). Further evolutionary research is indispensable for deciding which of them is more cogent.

\end{artengenv}

\begin{newrevplenv}{Robert Piechowicz}
	{\textit{Homo} (nie do końca) \textit{sapiens}}
	{\textit{Homo} (nie do końca) \textit{sapiens}}
	{\textit{Homo} (nie do końca) \textit{sapiens}}
	{Uniwersytet Papieski Jana Pawła II w Krakowie}
	{Adam Hart, \textit{Niedostosowani. Dlaczego ewolucja nie nadąża}, Wydawnictwo Uniwersytetu Jagiellońskiego, seria \#Nauka, Kraków 2021, s.~334.}



\lettrine[loversize=0.13,lines=2,lraise=-0.03,nindent=0em,findent=0.2pt]%
{A}{}dam Hart jest z~wykształcenia entomologiem, wykładającym na Uniwersytecie Gloucestershire komunikację naukową. Oprócz działalności naukowo-dydaktycznej tworzy audycje popularnonaukowe dla BBC Radio 4 i~BBC World Service. Swoją naukową i~popularyzatorską pasję realizuje także jako jeden z~narratorów bardzo popularnych cykli, jak \textit{Planet Ant: Life Inside the Colony} (BBC 4), \textit{Life on Planet Ant} (BBC 2) i~\textit{Hive Alive} (BBC 2). Oprócz działalności ściśle naukowej, której rezultatem jest ponad sto publikacji, Adam Hart realizuje się także jako autor popularnonaukowy -- jego debiutem dla szerszego czytelnika była wydana w~2015 roku książka \textit{The Life of Poo} (\textit{Życie kupy}).

Refleksja dotyczącą statusu człowieka w~świecie, mimo swojej zacnej i~długiej tradycji, niewątpliwie jest ważna i~aktualna. Jej ważkość wynika z~tego, że to istoty ludzkie dokonują tej autorefleksji, natomiast aktualność jest konsekwencją niekumulatywności filozoficznych rozważań. Co więcej, burzliwy rozwój nauk przyrodniczych zmotywował filozofów w~ciągu ostatniego stulecia do zreformowania niektórych -- uznawanych za oczywiste -- konstatacji. Odpowiedzialne podejście do relacji pomiędzy filozofią a~naukami empirycznymi przynosi bowiem pozytywne konsekwencje zarówno dla jednej, jak i~drugiej dziedziny.

Książka Adama Harta jest dziełem filozofującego naukowca, który podjął wyzwanie opisania niektórych elementów ludzkiej kondycji w~świecie. Już od pierwszych stron książki czytelnik może zauważyć, że wypracowanie konsensusu z~opiniami głoszącymi, że ,,Obecny niesłychanie szybki rozwój naukowo-techniczny zawdzięczamy globalizacji nauki oraz wspomaganiu naszych mózgów przez komputery, w~coraz większym stopniu posługujące się sztuczną inteligencją. Lepsze niż obecnie zrozumienie interakcji między ewolucją biologiczną i~kulturową jest jednak niezbędne do wyjaśnienia \guillemotleft niesamowitych właściwości człowieka jako gatunku\guillemotright'' 
%\label{ref:RNDa4MSH6AWZ6}(Kozłowski, 2018, s.~172)
\parencite[][s.~172]{kozlowski_ewolucja_2018} %
 nie jest oczywiste. Czy bowiem czasy nam współczesne dają podstawy do takiego optymizmu? Według Harta ,,Istoty ludzkie są doprawdy zdumiewające [...] jeszcze tak niedawno staliśmy boso i~gapiliśmy się na Księżyc oczyma pełnymi zdumienia; dziś możemy wziąć do ręki odłamek księżycowej skały [...] żyjemy w~świecie od początku do końca stworzonym przez nas samych, odseparowani od środowiska naturalnego, zbyt łatwo zapominamy, że jeśli pozbawi się nas osłonki nowoczesnego życia, okażemy się zwykłymi gadającymi małpami'' 
%\label{ref:RNDdAatnmJHaj}(Hart, 2021, s.~9).
\parencite[][s.~9]{hart_niedostosowani_2021}. %
 Zatem tematem książki jest problem dysharmonii pomiędzy ewolucją gatunku \textit{Homo sapiens} a~skokiem cywilizacyjnym, będącym jego dziełem. Co więcej, Autor wyraźnie nie do końca zgadza się w~swojej argumentacji z~rozpowszechnioną -- także przez naukowców -- opinią, zgodnie z~którą ludzkie życie toczy się w~wyjątkowo komfortowych warunkach wypracowanych w~oparciu o~właściwą dla naszego gatunku inwencyjność.

Jak przystało na empiryka, Autor nie stara się przeprowadzać ogólnych analiz relacji ,,człowiek-świat'' i~w konsekwencji preferuje interesującą intelektualnie wycieczkę, której kierunek wyznaczają pojęcia ,,ewolucji'', ,,adaptacji'' czy też ,,postępu''. Z~tego względu pierwszy rozdział ,,Chodząca, gadająca małpa'' jest bardzo skondensowanym wykładem dotyczącym ewolucji gatunku \textit{Homo sapiens}. Czytelnik nie znajdzie w~nim żadnych wątków filozoficznych, niemniej w~kontekście reszty książki niezwykłe zwięzła rekapitulacja wiedzy o~drogach prowadzących do człowieka współczesnego jest wyjątkowo trafna. Dzięki niej bowiem zrozumiała jest wymowa kolejnych rozdziałów, które -- z~różnych punktów widzenia -- wyjaśniają wybór tytułu książki. Okazuje się bowiem, że znaczna liczba cech, które posiada współczesny człowiek rażąco odbiega od wymogów świata, który dla siebie stworzył co skutkuje -- wynikającymi z~niedostosowania -- problemami z~którymi musi się zmagać. Tych zaś jest mnóstwo zaczynając od otyłości i~nietolerancji pokarmowych.

,,Najnowsze tendencje do globalizacji i~,,homogenizacji'' ludzkiej kultury żywieniowej, oparte na wzorcach zachodnich, w~połączeniu z~niemal powszechną dostępnością oraz atrakcyjnością niektórych typów żywności, tworzą środowisko, które dla części nas jest po prostu toksyczne''
%\label{ref:RND1GhWRG1ieD}(Hart, 2021, s.~108).
\parencite[][s.~108]{hart_niedostosowani_2021}. %
 Po raz kolejny okazuje się, że życie we współczesnym świecie jest dla istot ludzkich nie lada wyzwaniem. Jak możemy się przekonać na kolejnych kilkudziesięciu stronach porady dotyczące sposobu na życie -- w~tym przypadku realizowane przez odpowiednie żywienie -- dostarczane przez specjalistów i~laików zazwyczaj nie są zgodne z~wypracowanymi przez pokolenia strategiami metabolicznymi oraz poziomem komfortu, który sobie aktualnie zapewniliśmy. Komfort dostępności do przetworzonego pokarmu skutkuje także bardzo skomplikowanymi relacjami z~fauną żyjącą w~ludzkim układzie pokarmowym, który wynika z~ewolucyjnego zapóźnienia pożytecznych dla nas bakterii.

Lektura kolejnych rozdziałów pokazuje, że nie tylko ludzki metabolizm niechętnie współgra ze światem, który uzyskał swój aktualny charakter dzięki twórczej działalności gatunku ludzkiego. Przede wszystkim przyczyną wielu chorób cywilizacyjnych -- zbierających znaczne śmiertelne żniwa -- jest stres, co stanowi kolejną egzemplifikację paradoksalności kondycji ludzkiej. Reakcja stresowa ukształtowała się bowiem jako mechanizm obronny i~dzięki niej gatunek ludzki (albo przynajmniej dostatecznie wielu jego przedstawicieli) miało szanse przetrwać w~niekoniecznie pokojowo nastawionym środowisku. W~czasach współczesnych naturalne czynniki zagrażające życiu i~zdrowiu ludzkiemu zminimalizowano i~w konsekwencji reakcje stresowe obróciły się przeciwko ludziom. Jest to cokolwiek wymowny kontrprzykład dla znanego powiedzenia o~konstruktywnej roli harmonii i~konsensusu.

Problem staje się natomiast niezwykle palący w~kontekście rozważań zawartych w~dwóch kolejnych rozdziałach (,,Niszczycielski potencjał sieci'' oraz ,,Wyjątkowo agresywny gatunek''). Autor wskazuje, że budowanie zgody ma istotne i~naturalne ograniczenia wynikające ze zbyt szybkiego przejścia od konieczności utrzymywania wymiany informacyjnej z~niewielkim gronem odbiorców do współcześnie dostępnej możliwości (a niekiedy konieczności) harmonizowania poglądów w~kontekście grup składających się z~tysięcy czy też milionów ludzi. Próba stworzenia konsensusu staje przed nie lada wymogiem uwzględnienia wielu, niekiedy subtelnych, różnic w~poglądach, preferencjach czy działaniach.

,,Ewolucja przystosowała nas do tworzenia i~utrzymywania niewielkich sieci społecznych, które są dla nas źródłem wsparcia i~przyjaźni, a~co za tym idzie -- również przewagi selekcyjnej [...] jednakże w~ciągu ostatniej dekady odtworzyliśmy internetowy świat, który stał się dla większości z~nas ważnym elementem realnego środowiska [...], ale interakcje, w~które można wejść nawet w~stosunków skromnych sieciach społecznych [...] ze względu na swoją skalę mogą bardzo szybko przerodzić się w~obsesję''
%\label{ref:RNDiuicy96nPJ}(Hart, 2021, s.~204n).
\parencite[][s.~204n]{hart_niedostosowani_2021}. %
 Jeśli uwzględnimy także kolejne ewolucyjne dziedzictwo odpowiedzialne za przetrwanie naszych przodków, jakim jest skłonność do działań agresywnych, nie powinniśmy się dziwić, dlaczego gatunek ludzki wykazuje się bardzo silnymi skłonnościami niszczycielskimi wymierzonymi w~otaczający go świat.

Niezbyt optymistyczny wydźwięk refleksji Harta dopełniają trzy ostatnie rozdziały, poświęcone skłonnościom ludzi do uzależnień, łatwowierności gatunku \textit{Homo sapiens} oraz egoizmowi jako zasadniczemu motorowi działań. Mimo wielu interesujących informacji zawartych w~końcowej części książki, czytelnik może odnieść wrażenie, że gdy Hart-naukowiec przeistacza się w~Harta-moralizatora, to siła jego argumentów nieco słabnie. Refleksja, jakiej dostarcza Autor, traci bowiem ogólny wymiar przechodząc niepostrzeżenie w~indywidualne opinie wypowiadane na mocy zbudowanego do danego miejsca lektury autorytetu. Jest to jednak drobny mankament, niejednokrotnie występujący w~publikacjach filozofujących naukowców, który nie powinien decydować o~lekturze całej książki. Niemniej warto zadać sobie pytanie, kto powinien zapoznać się z~książką \textit{Niedostosowani}? Pierwsza przychodząca na myśl odpowiedź, to każdy, kto chce zrozumieć doskwierające nam problemy związane z~niedostosowaniem do antroposfery. Hart podejmuje bowiem bardzo aktualne zagadnienie, przy czym przedstawia je na o~wiele bardziej profesjonalnym poziomie, niż ten, do którego mamy najłatwiejszy -- dzięki Internetowi -- dostęp. Czy jednak publikacja Harta będzie sukcesem wydawniczym? W~tym przypadku odpowiedź wydaje się mniej optymistyczna. Jak sądzę, po książkę tą sięgnie niewielu czytelników, a~lekturę ukończą nieliczni. Na zasadność pierwszej konkluzji wskazał sam Autor, dokonując krytycznej analizy wpływu mediów społecznościowych i~internetowych celebrytów
%\label{ref:RNDyK69eKl2d3}(zob. Hart, 2021, s.~179–180).
\parencite[zob.][s.~179–180]{hart_niedostosowani_2021}. %
 Poszukiwanie informacji, mimo zdumiewającej łatwości, współcześnie uwarunkowane jest czasowym i~światopoglądowym komfortem. Dlaczego bowiem mielibyśmy szukać informacji rzetelnych, choć niekoniecznie łatwych w~odbiorze, skoro możemy sięgnąć do danych spreparowanych tak, abyśmy nie musieli zbytnio się wysilać?

Jakże bezcelową wydaje się lektura 306 stron książki, skoro omawiane problemy na pewno ktoś poruszył w~,,sieci''. Problem tego, ,,jak odróżnić treści merytorycznie wartościowe od tych mniej zaawansowanych czy wręcz takich, które mogą wprowadzać w~błąd?'' jest współcześnie dobrze znany. Co więcej, ,,Wydaje się, że jest to jeden tych problemów, dla których trudno znaleźć satysfakcjonujące rozwiązanie, niegodzące w~tak ważną w~przestrzeni internetowej swobodę wypowiedzi''
%\label{ref:RNDQbHrJjmJnD}(Polak i~in., 2021, s.~255).
\parencite[][s.~255]{polak_internetowe_2021}. %
 Niestety, Hart nie pomaga Czytelnikowi w~lekturze, gdyż omawiane problemy traktuje bardzo poważnie i~nie trywializuje stosowanej argumentacji. Dlatego zapoznanie się z~treścią jego książki wymaga intelektualnej pokory, na którą -- niestety -- wielu z~nas nie stać. Bez wątpienia bowiem Autor podjął interesujące i~ważne wyzwanie, które powinno stać się inspiracją dla odpowiedzialnej filozoficznej refleksji dotyczącej kondycji współczesnych nam przedstawicieli gatunku ludzkiego. Jest to o~tyle ważne, że filozofowie mają naturalną tendencję do zamykania się w~dobrze ugruntowanej w~tradycji refleksji na dany temat, ignorując niejednokrotnie nowe, a~zarazem niekiedy niezgodne z~ich przekonaniami dane.



%-------------------------




\selectlanguage{english}
\vspace{5mm}%
\begin{flushright}
{\chaptitleeng\color{black!50}{\textit{Homo} (not really) \textit{sapiens}}}
\end{flushright}

%\vspace{10mm}%
{\subsubsectit{\hfill Abstract}}\\
{For millennia, philosophers have pondered the question, ``Who are we?'' It becomes a burning question at an age of scientific and technological civilization and a world full of conveniences, mainly because human life is still not comfortable. Adam Hart takes up the challenge of confronting the evolutionary
achievements of the \textit{Homo} species and the environment it has produced, showing that our life---for
fundamental reasons---cannot be the realization of idyllic visions.
}\par%
\vspace{2mm}%
{\subsubsectit{\hfill Keywords}}\\%
{philosophy, evolution, culture, contemporary world.}%

\selectlanguage{polish}




\end{newrevplenv}
